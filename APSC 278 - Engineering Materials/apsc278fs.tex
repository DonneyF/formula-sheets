% !Tex program = pdflatex

\documentclass[12pt,landscape]{article}
\usepackage{multicol}
\usepackage{calc}
\usepackage{ifthen}
\usepackage[landscape]{geometry}
\usepackage{amsmath,amsthm,amsfonts,amssymb}
\usepackage{color,graphicx,overpic}
\usepackage{hyperref}
\usepackage{enumitem}
\usepackage{upgreek}
\usepackage[italicdiff]{physics}
\usepackage{newtxtext,newtxmath}
\usepackage{mdframed}
\usepackage{amsbsy}

% This sets page margins to .5 inch if using letter paper, and to 1cm
% if using A4 paper. (This probably isn't strictly necessary.)
% If using another size paper, use default 1cm margins.
\ifthenelse{\lengthtest { \paperwidth = 11in}}
	{ \geometry{top=.5in,left=.5in,right=.5in,bottom=.5in} }
	{\ifthenelse{ \lengthtest{ \paperwidth = 297mm}}
		{\geometry{top=1cm,left=1cm,right=1cm,bottom=1cm} }
		{\geometry{top=1cm,left=1cm,right=1cm,bottom=1cm} }
	}

% Turn off header and footer
\pagestyle{empty}
 

% Redefine section commands to use less space
\makeatletter
\renewcommand{\section}{\@startsection{section}{1}{0mm}%
                                {-1ex plus -.5ex minus -.2ex}%
                                {0.5ex plus .2ex}%x
                                {\normalfont\normalsize\bfseries}}
\renewcommand{\subsection}{\@startsection{subsection}{2}{0mm}%
                                {-1explus -.5ex minus -.2ex}%
                                {0.5ex plus .2ex}%
                                {\normalfont\small\bfseries}}
\renewcommand{\subsubsection}{\@startsection{subsubsection}{3}{0mm}%
                                {-1ex plus -.5ex minus -.2ex}%
                                {1ex plus .2ex}%
                                {\normalfont\footnotessize\bfseries}}
\makeatother

% Define BibTeX command
\def\BibTeX{{\rm B\kern-.05em{\sc i\kern-.025em b}\kern-.08em
    T\kern-.1667em\lower.7ex\hbox{E}\kern-.125emX}}

% Don't print section numbers
\setcounter{secnumdepth}{0}


\setlength{\parindent}{0pt}
\setlength{\parskip}{1pt plus 0.5ex}

\newcommand{\tab}{\hspace*{1em}}
\newcommand{\ds}{\displaystyle}

% Redefine some commands for newtxmath boldness
\renewcommand{\grad}{\nabla}
\renewcommand{\curl}[1]{\nabla\times#1}
\renewcommand{\div}[1]{\nabla\cdot#1}
\renewcommand{\cross}{\times}
\newcommand{\defn}[1]{\textbf{Def} (\emph{#1})}
\newcommand{\thm}[1]{\textbf{Thm} (\emph{#1})}

\newcommand{\Var}[1]{\mathrm{Var}(#1)}
\newcommand{\Cov}[1]{\mathrm{Cov}(#1)}

\mdfsetup{skipabove=2pt,skipbelow=2pt, innertopmargin=-6pt, innerbottommargin=2pt, innerleftmargin=2pt, innerrightmargin=2pt}
\theoremstyle{definition}
\newmdtheoremenv{theorem}{Theorem}

% -----------------------------------------------------------------------

\begin{document}

\raggedright
\footnotesize
\begin{multicols*}{3}

\raggedcolumns

% multicol parameters
% These lengths are set only within the two main columns
%\setlength{\columnseprule}{0.25pt}
\setlength{\premulticols}{1pt}
\setlength{\postmulticols}{1pt}
\setlength{\multicolsep}{1pt}
\setlength{\columnsep}{2pt}

\begin{center}
	\Large{\underline{APSC 278 Formula Sheet}}
\end{center}

\section{Bonding and Properties}
Equilibrium at $r_0$:\\
\tab $F_\text{net} = \dv{E}{r} = 0$

Thermal Expansion:\\
\tab $\Delta L = L_0 \alpha \Delta T$

\section{Mechanical Properties}
Hooke's Law:\\
\tab $\sigma = E \epsilon$

Poisson's Ratio:\\
\tab $\ds \nu = - \frac{\epsilon_x}{\epsilon_z} = -\frac{\epsilon_y}{\epsilon_z}$

Shear Modulus:\\
\tab $\ds G = \frac{\tau}{\gamma} = \frac{F}{A_0 \tan\theta}$

\tab $E = 2G(1+\nu)$

Toughness:\\
\tab $\ds \text{Toughness} = \int_{0}^{\epsilon_f}\sigma\mathop{d\epsilon}\approx \frac{\sigma_\text{YS} + \sigma_\text{UTS}}{2}\epsilon_f $

Modulus of Resilience:\\
\tab $\ds U_r = \int_{0}^{\epsilon_Y}\sigma\mathop{d\epsilon} \approx \frac{\sigma_\text{YS}^2}{2E}$

Work Hardening:\\
\tab $\sigma_T = K \epsilon_T^n$

True Stress and Strain:\\
\tab $\ds \epsilon_\text{true} = \int_{L_0}^{L_i}\frac{dL}{L} = \ln(\epsilon_\text{Eng} + 1)$

\tab $\sigma_\text{true} = \sigma_\text{eng}(\epsilon_\text{eng} + 1)$

\section{Crystal Structures}
\bgroup
\def\arraystretch{1.2}%
\tab \begin{tabular}{|l|c | c | c |} 
	\hline
	& \begin{tabular}[x]{@{}c@{}}Lattice\\Parameter ($a$)\end{tabular} & \begin{tabular}[x]{@{}c@{}}Atoms per\\unit cell\end{tabular} & APF\\
	\hline
	BCC & $\frac{4\sqrt{3}}{3} R$ & 2  & 0.68\\
	\hline
	FCC & $2\sqrt{2}R$ & 4  & 0.74\\
	\hline
	HCP & -  & 6  & 0.74\\
	\hline
\end{tabular}
\egroup

Density (g/cm$^3$):\\
\tab $\ds \rho_\text{Th} = \frac{NMW_i}{V_c N_A}$\\
\vspace{2mm}
\tab $n$ = number of atoms per unit cell\\
\tab $MW_i$ = Atomic weight (g/mol)\\
\tab $V_c$ = unit cell volume\\
\tab $N_A$ = Avogadro's Number ($6.023 \cdot 10^{23}$ atoms/mol)

\section{Effect of Temperature on Deformation}
Homologous Temperature:\\
\tab $\ds T_H = \frac{T_\text{deformation}}{T_\text{melt}}$

Homologous Temperature for pure metals:\\
\tab $\begin{cases}
\text{Cold working} & T_H < 0.4\\
\text{Hot working} & T_H > 0.4
\end{cases}$

Homologous Temperature for alloys:\\
\tab $\begin{cases}
\text{Cold working} & T_H < 0.6\\
\text{Hot working} & T_H > 0.6
\end{cases}$

Cold Working (rolling):\\
\tab \% cold work = $\ds \frac{t_0 - t_f}{t_0} \times 100\%$

Recrystallization rate:\\
\tab rate $\ds = A\exp(-\frac{Q}{RT})$\\
\tab $t_\text{recrx} = 1/\text{rate}$\\
\tab $Q = $ thermal activation energy

Grain Growth:\\
\tab $D(t) = D_0t^m$

Hall-Petch Equation for grain size $d$:\\
\tab $\sigma_\text{YS} = \sigma_0 + k_y d^{-1/2}$

Creep rate:\\
\tab $\ds \dot{\epsilon}_{c,ss} = K_2\sigma^n\exp(-\frac{Q_c}{RT})$

Larson Miller parameter ($C \approx 20$):\\
\tab $m = T(C + \log_{10}(t_r))$

\section{Fracture}

Stress Concentration Factor for an elliptical notch:\\
\tab $\ds \sigma_m = \sigma_\text{net} \left(1 + 2 \sqrt{\frac{a}{r_t}}\right) = \sigma_\text{net}K_t$

Griffith Equation for surface energy $\gamma_s$:\\
\tab $\ds \sigma_\text{critical} = \sqrt{\frac{2E\gamma_s}{\pi a_c}}$

Griffith Equation extension to ductile materials:\\
\tab $\ds \sigma_\text{critical} = \sqrt{\frac{2EG_c}{\pi a_c}}$\\
\tab $G_c$ = critical strain  energy release rate

Stress Intensity Factor:\\
\tab $K = Y\sigma\sqrt{\pi a_c}$

Paris Equation (crack growth in steady state creep):\\
\tab $\ds \dv{a}{N} = C(\Delta K)^N$

\section{Composites}
Isostrain:\\
\tab $\epsilon_c = \epsilon_m = \epsilon_f$\\
\tab $F_c = F_f + F_m$\\
\tab $E_c = E_ff_f + E_mf_m$\\
\tab $\ds \frac{F_f}{F_c} = \frac{f_f}{f_f + \frac{E_m}{E_f}f_m}$

Isostress:\\
\tab $\sigma_c = \sigma_m = \sigma_f$\\
\tab $\epsilon_c = \epsilon_m f_m + \epsilon_f f_f$\\
\tab $\ds E_c = \frac{E_m E_f}{E_mf_f + E_f f_m}$\\

\section{Electrical Properties}
Resistance in a straight conductor:\\
\tab $\ds R = \frac{\rho l}{A}$

Transmission Line Sag:\\
\tab $\ds \delta \approx \frac{wl^2}{8H}$\\
\tab $L = l + \frac{8\delta^2}{3l}$\\
\tab $L_\text{total} = L(1 + \epsilon_\sigma + \epsilon_T)$

Conductivity in metals and semiconductors:\\
\tab $\sigma = n\abs{e}\mu_e + p\abs{e}\mu_p$\\
\tab For metals, $p = 0$. For intrinsic semiconductors, $n = p$.

% Footer content
\rule{0.3\linewidth}{0.25pt}
\scriptsize\\
Updated \today\\
\href{https://github.com/DonneyF/formula-sheets}{https://github.com/DonneyF/formula-sheets}
\end{multicols*}%

\end{document}
