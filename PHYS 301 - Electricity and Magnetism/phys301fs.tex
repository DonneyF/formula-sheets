\documentclass[12pt,landscape]{article}
\usepackage{multicol}
\usepackage{calc}
\usepackage{ifthen}
\usepackage[landscape]{geometry}
\usepackage{amsmath,amsthm,amsfonts,amssymb}
\usepackage{color,graphicx,overpic}
\usepackage{hyperref}
\usepackage{enumitem}
\usepackage{upgreek}
\usepackage{physics}
\usepackage{newtxtext,newtxmath}

% This sets page margins to .5 inch if using letter paper, and to 1cm
% if using A4 paper. (This probably isn't strictly necessary.)
% If using another size paper, use default 1cm margins.
\ifthenelse{\lengthtest { \paperwidth = 11in}}
	{ \geometry{top=.5in,left=.5in,right=.5in,bottom=.5in} }
	{\ifthenelse{ \lengthtest{ \paperwidth = 297mm}}
		{\geometry{top=1cm,left=1cm,right=1cm,bottom=1cm} }
		{\geometry{top=1cm,left=1cm,right=1cm,bottom=1cm} }
	}

% Turn off header and footer
\pagestyle{empty}
 

% Redefine section commands to use less space
\makeatletter
\renewcommand{\section}{\@startsection{section}{1}{0mm}%
                                {-1ex plus -.5ex minus -.2ex}%
                                {0.5ex plus .2ex}%x
                                {\normalfont\normalsize\bfseries}}
\renewcommand{\subsection}{\@startsection{subsection}{2}{0mm}%
                                {-1explus -.5ex minus -.2ex}%
                                {0.5ex plus .2ex}%
                                {\normalfont\small\bfseries}}
\renewcommand{\subsubsection}{\@startsection{subsubsection}{3}{0mm}%
                                {-1ex plus -.5ex minus -.2ex}%
                                {1ex plus .2ex}%
                                {\normalfont\footnotessize\bfseries}}
\makeatother

% Define BibTeX command
\def\BibTeX{{\rm B\kern-.05em{\sc i\kern-.025em b}\kern-.08em
    T\kern-.1667em\lower.7ex\hbox{E}\kern-.125emX}}

% Don't print section numbers
\setcounter{secnumdepth}{0}


\setlength{\parindent}{0pt}
\setlength{\parskip}{1pt plus 0.5ex}

\newcommand{\tab}{\hspace{.02\textwidth}}
\newcommand{\ds}{\displaystyle}

\def\rcurs{{\mbox{$\resizebox{.09in}{.08in}{\includegraphics[trim= 1em 0 14em 0,clip]{ScriptR}}$}}}
\def\brcurs{{\mbox{$\resizebox{.09in}{.08in}{\includegraphics[trim= 1em 0 14em 0,clip]{BoldR}}$}}}

\renewcommand{\dv}[2]{\frac{d#1}{d#2}}

% Redefine some commands for newtxmath boldness
\renewcommand{\grad}{\nabla}
\renewcommand{\curl}[1]{\nabla\times#1}
\renewcommand{\div}[1]{\nabla\cdot#1}
\renewcommand{\cross}{\times}

% -----------------------------------------------------------------------

\begin{document}

\raggedright
\footnotesize
\begin{multicols}{3}


% multicol parameters
% These lengths are set only within the two main columns
%\setlength{\columnseprule}{0.25pt}
\setlength{\premulticols}{1pt}
\setlength{\postmulticols}{1pt}
\setlength{\multicolsep}{1pt}
\setlength{\columnsep}{2pt}

\begin{center}
	\Large{\underline{PHYS 301 Formula Sheet}}
\end{center}

\section{Differential Maxwell's Equations}
\hspace{3mm}\begin{tabular}{p{2cm}p{4cm}}
	$\ds \div{\vb{E}} = \frac{\rho}{\epsilon_0}$ & $\ds \curl{\vb{E}} = -\pdv{\vb{B}}{t}$\\
	$\ds \div{\vb{B}} = 0$ & $\ds \curl{\vb{B}} = \mu_0\vb{J} + \mu_0\epsilon_0\pdv{\vb{E}}{t}$
\end{tabular}

\section{Integral Maxwell's Equations}

\begin{tabular}{p{2.75cm}p{5cm}}
	$\ds \oint \vb{E}\cdot d\vb{a} = \frac{Q_\text{enc}}{\epsilon_0}$ & $\ds \oint \vb{B}\cdot d\vb{a} = 0$\\
	$\ds \oint \vb{E}\cdot d\vb{l} = -\dv{\Phi_B}{t}$ & $\ds \oint \vb{B}\cdot d\vb{l} = \mu_0 I_\text{enc} + \epsilon_0\mu_0 \dv{\Phi_E}{t}$
\end{tabular}
\section{Electrostatics}
\subsection{The Electric Field}
Coulomb's Law:\\
\tab $\ds \vb{F} = \frac{1}{4\pi\epsilon_0} \frac{qQ}{\rcurs} \vu{\brcurs}$

The Electric Field:\\
\tab $\vb{F} = Q \vb{E}$ \qquad $\vb{E} = -\grad{V}$

Electric Field due to discrete point charges:\\
\tab $\ds \vb{E(\vb{r})} = \frac{1}{4\pi\epsilon_0} \sum_{i = 1}^{n} \frac{q_i}{\rcurs_i^2} \vu{\brcurs}_i$

Electric Field due to a continuous charge distribution:\\
\tab $\ds \vb{E(\vb{r})} = \frac{1}{4\pi\epsilon_0} \int \frac{\vu{\brcurs}}{\rcurs^2}\,dq$

\subsection{Electric Potential}
\tab $\ds V(b) - V(a) = - \int_{a}^{b} \vb{E} \cdot d\vb{l}$

Poisson's Equation:\\
\tab $\ds \nabla^2 V = \rho/\epsilon_0$

Potential due to a localized charge distribution:\\
\tab $\ds V(\vb{r}) = \frac{1}{4\pi\epsilon_0} \int \frac{\rho(\vb{r'})}{\rcurs}\,dV$

\subsection{Work and Energy in Electrostatics}
Energy stored in a point charge distribution:\\
\tab $\ds W = \frac{1}{2} \sum_{i=1}^{n} q_i V(\vb{r}_i)$

Energy of a continuous charge distribution:\\
\tab $\ds W = \frac{1}{2} \int \rho V\,d\uptau$

Total energy of a continuous charge distribution:\\
\tab $\ds W = \frac{\epsilon_0}{2} \int E^2\,d\uptau$ \quad (all space)

\subsection{Conductors}
\tab $\vb{E} = \vb{0}$ inside a conductor.\\

Electric Field immediately outside a conductor:\\
\tab $\ds \vb{E} = \frac{\sigma}{\epsilon_0} \vu{\vb{n}}$

Surface charge:\\
\tab $\ds \sigma = -\epsilon_0 \pdv{V}{n}$

Capacitors:\\
\tab $Q = CV$ 	
\\
\tab $\ds W = \frac{1}{2}CV^2 = \frac{1}{2}\frac{Q^2}{C} = \frac{1}{2}QV$

\section{Potentials}
Laplace's Equation:\\
\tab $\ds \nabla^2 V = \pdv[2]{V}{x} + \pdv[2]{V}{y} + \pdv[2]{V}{z}= 0$

\subsection{Separation of Variables}
\begin{equation*}
	\hspace{-1cm}\int_0^a \sin(\frac{n\pi x}{a})\sin(\frac{m\pi x}{a})\,dx = 
	\begin{cases}
	0 & \text{if } n \neq m\\
	a/2 & \text{if } n = m
	\end{cases}
\end{equation*}

Legendre Polynomials:
\begin{itemize}
	\itemsep0em
	\item $P_0(x) = 1$
	\item $P_1(x) = x$
	\item $P_2(x) = (3x^2 - 1)/2$
	\item $P_3(x) = (5x^3 - 3x)/2$
\end{itemize}

Solution to Laplace in spherical ($\phi$ independent):\\
\tab $\ds V(r,\theta) = \sum_{l = 0}^{\infty}\left(A_l r^l + \frac{B_l}{r^{l + 1}}\right)P_l(\cos\theta)$

Solution to Laplace in cylindrical ($z$ independent):\\
\tab $\ds V(s,\phi) = A_0\ln(s) + B_0 +$\\\tab$\ds \sum_{n = 1}^{\infty}(A_n s^{-n} + B_ns^{-n})(C_n\cos(n\phi) + D_n\sin(n\phi))$

\subsection{Multipole Expansion}
Potential at large distances ($\alpha$ is between $\vb{r}$ and $\vb{r'}$):\\
\tab $\ds V(\vb{r}) = \frac{1}{4\pi\epsilon_0}\sum_{n = 0}^{\infty}\frac{1}{r^{n + 1}}\int (r')^n P_n(\cos\alpha)\rho(\vb{r'})\,d\uptau$

Dipole Moment:\\
\tab $\ds \vb{p} = \sum_{i=1}^{n}q_i \vb{r'}_i = \int \vb{r'}\rho(\vb{r'})\,d\uptau$

Electric Dipole Potential:\\
\tab $\ds V_{\text{dip}}(\vb{r}) = \frac{1}{4\pi\epsilon_0} \frac{\vb{p}\cdot\vu{r}}{r^2}$

\section{Electric Fields in Matter}

Bound Charges:\\
\tab $\sigma_b = \vb{P}\cdot \vu{n} \qquad \rho_b = - \div{\vb{P}}$

The Electric Displacement:\\
\tab $\vb{D} = \epsilon_0\vb{E} + \vb{P}$ \quad\, $\div{\vb{D}} = \rho_f$ \quad\, $\ds \oint \vb{D} \cdot d\vb{a} = Q_{f,\text{enc}}$

\subsection{Linear Dielectrics}
Polarization:\\
\tab $\vb{P} = \epsilon_0\chi_e \vb{E}$

Electric Displacement:\\
\tab $\vb{D} = \epsilon_0(1 + \chi_e)\vb{E} = \epsilon\vb{E}$

Energy in a Dielectric System:\\
\tab $\ds W = \frac{1}{2}\int \vb{D}\cdot\vb{E}\,d\uptau$

\subsection{Boundary Conditions in Electrostatics}
\begin{itemize}
	\itemsep0em
	\item $\vb{D}_{\text{above}}^{\perp} - \vb{D}_{\text{below}}^{\perp}  = \sigma_f$
	\item $\vb{D}_{\text{above}}^{\parallel} - \vb{D}_{\text{below}}^{\parallel} = \vb{P}_{\text{above}}^{\parallel} - \vb{P}_{\text{below}}^{\parallel}$
	\item $\vb{E}_{\text{above}}^{\perp} - \vb{E}_{\text{below}}^{\perp}  = \sigma/\epsilon_0$
	\item $\vb{E}_{\text{above}}^{\parallel} - \vb{E}_{\text{below}}^{\parallel}  = \vb{0}$
	\item $V_\text{above} = V_\text{below}$
	\item $\epsilon_\text{above}\vb{E}_\text{above}^\perp - \epsilon_\text{below}\vb{E}_\text{below}^\perp = \sigma_f$
	\item $\ds \epsilon_\text{above}\pdv{V_\text{above}}{n} - \epsilon_\text{below}\pdv{V_\text{below}}{n} = \sigma_f$
\end{itemize}

\section{Magnetostatics}
Lorentz Force Law:\\
\tab $\vb{F} = q(\vb{E}+\vb{v}\cross\vb{B})$

Currents:\\
\tab $\vb{I} = \lambda \vb{v}$ \qquad $\vb{K} = \sigma\vb{v}$ \qquad $\vb{J} = \rho\vb{v}$

Biot-Savart Law:\\
\tab $\ds \vb{B}(\vb{r}) = \frac{\mu_0}{4\pi}\int \frac{\vb{I}\cross\vu{\brcurs}}{\rcurs^2}\,dl' = \frac{\mu_0I}{4\pi}\int \frac{d\vb{l'}\cross\vu{\brcurs}}{\rcurs^2}$

\subsection{Magnetic Vector Potential}
\tab $\vb{B} = \curl{\vb{A}}$ \quad where \quad $\div{\vb{A}} = 0$

Vector Potential Poisson's Equation:\\
\tab $\nabla^2\vb{A} = -\mu_0 \vb{J}$

Vector Potential when $\vb{A} \rightarrow \vb{0}$ at infinity:\\
\tab $\ds \vb{A}(\vb{r}) = \frac{\mu_0}{4\pi}\int \frac{\vb{J}(\vb{r'})}{\rcurs}\,d\uptau$

Multipole Expansion of a current loop:\\
\tab $\ds \vb{A}(\vb{r}) = \frac{\mu_0}{4\pi} \sum_{n = 0}^{\infty} \frac{1}{r^{n+1}} \oint (r')^n P_n(\cos\alpha)\,d\vb{l'}$\\
\tab $\ds \vb{A}(\vb{r}) = \frac{\mu_0}{4\pi} \frac{\vb{m}\times\vu{r}}{r^2}$

Magnetic Dipole Moment for a vector area $\vb{a}$:\\
\tab $\ds \vb{m} = I\int d\vb{A} = I\vb{a}$

\section{Magnetic Fields in Matter}
Bound Currents:\\
\tab $\vb{J}_B = \curl{\vb{M}}$ \qquad $\vb{K}_B = \vb{M}\cross\vb{n}$

Auxillary Field:\\
\tab $\ds \vb{H} = \frac{\vb{B}}{\mu_0} - \vb{M}$ \quad\, $\curl{\vb{H}} = \vb{J}_f$ \quad\, $\ds \oint \vb{H}\cdot d\vb{l} = I_\text{free}$

\subsection{Linear Media}
Magnetization in linear media:\\
\tab $\vb{M} = \chi_m \vb{H}$

Auxillary Field:\\
\tab $\vb{B} = \mu_0(\vb{H} + \vb{M}) = \mu_0(1 + \chi_m)\vb{H} = \mu \vb{H}$

Volume bound current:\\
\tab $\vb{J}_B = \chi_m \vb{J}_f$


\subsection{Boundary Conditions in Magnetostatics}
\begin{itemize}
	\itemsep0em
	\item $\vb{B}_\text{above}^\parallel - \vb{B}_\text{below}^\parallel = \mu_0\vb{K}$
	\item $\vb{B}_\text{above} - \vb{B}_\text{below} = \mu_0(\vb{K}\cross\vu{n})$
	\item $\vb{A}_\text{above} = \vb{A}_\text{below}$
	\item $\ds \pdv{A_\text{above}}{n} - \pdv{A_\text{below}}{n} = -\mu_0\vb{K}$
	\item $\vb{H}_\text{above}^\perp - \vb{H}_\text{below}^\perp = - (\vb{M}_\text{above}^\perp - \vb{M}_\text{below}^\perp) $
	\item $\vb{H}_\text{above}^\parallel - \vb{H}_\text{below}^\parallel = \vb{K}_f \cross \vu{n}$
\end{itemize}

\section{Vector Derivatives}
\subsection{Cartesian}
$d\vb{l} = dx\,\vu{x} + dy\,\vu{y} + dz\,\vu{z}$
\hspace{1cm}
$d\uptau = dx\,dy\,dz$

Gradient:\\
\tab $\ds \grad{f} =  \pdv{f}{x}\vu{x} + \pdv{f}{y}\vu{y} + \pdv{f}{z}\vu{z}$

Divergence:\\
\tab $\ds \div{\vb{v}} = \pdv{v_x}{x} + \pdv{v_y}{y} + \pdv{v_z}{z}$

Curl:\\
\vspace{-3mm}
\tab $\ds \curl{\vb{v}} = \left(\pdv{v_z}{y} - \pdv{v_y}{z}\right)\vu{x} + \left(\pdv{v_x}{z} - \pdv{v_z}{x}\right)\vu{y} + \left(\pdv{v_y}{x} - \pdv{v_x}{y}\right)\vu{z}$

\subsection{Spherical}
$d\uptau = r^2\sin\theta\,dr\,d\theta\,d\phi$

Gradient:\\
\tab $\ds \grad{f} = \pdv{f}{r}\vu{r} + \frac{1}{r}\pdv{f}{\theta}\vu*{\theta} + \frac{1}{r\sin\theta}\pdv{f}{\phi}\vu*{\phi}$

Divergence:\\
\tab $\ds \div{\vb{v}} = \frac{1}{r^2}\pdv{r}(r^2v_r) + \frac{1}{r\sin\theta}\pdv{\theta}(\sin\theta v_\theta) + \frac{1}{r\sin\theta}\pdv{v_\phi}{\phi}$

Curl:\\
\tab $\ds \curl{\vb{v}} = \frac{1}{r\sin\theta}\left[\pdv{\theta}(\sin\theta\, v_\phi)- \pdv{v_\theta}{\phi}\right]\vu{r}\,+$\\
\tab \tab $\ds \frac{1}{r}\left[\frac{1}{\sin\theta}\pdv{v_r}{\phi}-\pdv{r}(r v_\phi)\right]\vu*{\theta} + \frac{1}{r}\left[\pdv{r}(r v_\theta)-\pdv{v_r}{\theta}\right]\vu*{\phi}$

\subsection{Cylindrical}
$d\uptau = s\,ds\,d\phi\,dz$

Gradient:\\
\tab $\ds \grad{f} = \pdv{f}{s}\vu{s}+\frac{1}{s}\pdv{f}{\phi}\vu*{\phi}+\pdv{f}{z}\vu{z}$

Divergence:\\
\tab $\ds \div{\vb{v}} = \frac{1}{s}\pdv{s}(sv_s)+\frac{1}{s}\pdv{v_\phi}{\phi} + \pdv{v_z}{z}$

Curl:\\
\tab $\ds \curl{\vb{v}} = \left[\frac{1}{s}\pdv{v_z}{\phi}-\pdv{v_\phi}{z}\right]\vu{s}+\left[\pdv{v_s}{z}-\pdv{v_z}{s}\right]\vu*{\phi}+\frac{1}{s}\left[\pdv{s}(sv_\phi)-\pdv{v_s}{\phi}\right]\vu{z}$

\section{Fundamental Theorems}
Fundamental Theorem of Line Integrals:\\
\tab $\ds \int_{\vb{a}}^{\vb{b}}(\grad{f})\cdot d\vb{l} = f(\vb{b})-f(\vb{a})$

Divergence Theorem:\\
\tab $\ds \int (\div \vb{A})\, d\uptau = \oint \vb{A}\cdot d\vb{a}$

Stoke's Theorem:\\
\tab $\ds \int (\curl{\vb{A}})\cdot d\vb{a} = \oint \vb{A}\cdot d\vb{l}$

\section{Vector Identities}
\tab $\ds \div{\left(\frac{\vu{\brcurs}}{\rcurs^2}\right)} = 4\pi \delta^3(\brcurs)$
\\
\tab $\ds \grad{\left(\frac{1}{\rcurs}\right)} = -\frac{\vu{\brcurs}}{\rcurs}$
\\
\tab $\ds \delta(kx) = \frac{1}{\abs{k}}\delta(x)$

\section{Spherical Coordinates}
\tab $x = r\sin\theta\cos\phi$
\\
\tab $y = r\sin\theta\sin\phi$
\\
\tab $z = r\cos\theta$
\\
\vspace{3mm}
\tab $\vu{x} = \sin\theta\cos\phi\,\vu{r} + \cos\theta\cos\phi\,\vu*{\theta} - \sin\phi\,\vu*{\phi}$
\\
\tab $\vu{y} = \sin\theta\sin\phi\,\vu{r} + \cos\theta\sin\phi\,\vu*{\theta} + \cos\phi\,\vu*{\phi}$
\\
\tab $\vu{z} = \cos\theta \, \vu{r} - \sin\theta \, \vu*{\theta}$
\\
\vspace{3mm}
\tab $r=\sqrt{x^2 + y^2 + z^2}$
\\
\tab $\theta = \tan^{-1}\left(\sqrt{x^2+y^2}/z \right)$
\\
\tab $\phi = \tan^{-1}(y/x)$
\\
\vspace{3mm}
\tab $\vu{r}=\sin\theta\cos\phi\,\vu{x} + \sin\theta\sin\phi\,\vu{y}+\cos\theta\,\vu{z}$
\\
\tab $\vu*{\theta}=\cos\theta\cos\phi\,\vu{x}+\cos\theta\sin\phi\,\vu{y}-\sin\theta\,\vu{z}$
\\
\tab $\vu*{\phi} = -\sin\phi\,\vu{x}+\cos\phi\,\vu{y}$

\section{Cylindrical Coordinates}
\tab $x = s\cos\phi$
\\
\tab $y = s \sin\phi$
\\
\tab $z=z$
\\
\vspace{3mm}
\tab $\vu{x}=\cos\phi\,\vu{s}-\sin\phi\vu*{\phi}$
\\
\tab $\vu{y} = \sin\phi\,\vu{s}+\cos\phi\,\vu*{\phi}$
\\
\tab $\vu{z} = \vu{z}$
\\
\vspace{3mm}
\tab $s = \sqrt{x^2+y^2}$
\\
\tab $\phi = \tan^{-1}(y/x)$
\\
\tab $z = z$
\\
\vspace{3mm}
\tab $\vu{s} = \cos\phi\,\vu{x}+\sin\phi\,\vu{y}$
\\
\tab $\vu*{\phi}=-\sin\phi\,\vu{x}+\cos\phi\,\vu{y}$
\\
\tab $\vu{z} = \vu{z}$

% Footer content
\rule{0.3\linewidth}{0.25pt}
\scriptsize\\
Updated \today\\
\href{https://github.com/DonneyF/formula-sheets}{https://github.com/DonneyF/formula-sheets}
\end{multicols}
\end{document}
