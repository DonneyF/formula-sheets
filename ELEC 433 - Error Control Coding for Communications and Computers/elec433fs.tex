\documentclass[12pt,landscape]{article}
\usepackage{multicol}
\usepackage{calc}
\usepackage{ifthen}
\usepackage[landscape]{geometry}
\usepackage{amsmath,amsthm,amsfonts,amssymb}
\usepackage{color,graphicx,overpic}
\usepackage{hyperref}
\usepackage{enumitem}
\usepackage{upgreek}
\usepackage{physics}
\usepackage{newtxtext,newtxmath}
\usepackage{bm}

% This sets page margins to .5 inch if using letter paper, and to 1cm
% if using A4 paper. (This probably isn't strictly necessary.)
% If using another size paper, use default 1cm margins.
\ifthenelse{\lengthtest { \paperwidth = 11in}}
	{ \geometry{top=.5in,left=.5in,right=.5in,bottom=.5in} }
	{\ifthenelse{ \lengthtest{ \paperwidth = 297mm}}
		{\geometry{top=1cm,left=1cm,right=1cm,bottom=1cm} }
		{\geometry{top=1cm,left=1cm,right=1cm,bottom=1cm} }
	}

% Turn off header and footer
\pagestyle{empty}
 

% Redefine section commands to use less space
\makeatletter
\renewcommand{\section}{\@startsection{section}{1}{0mm}%
                                {-1ex plus -.5ex minus -.2ex}%
                                {0.5ex plus .2ex}%x
                                {\normalfont\normalsize\bfseries}}
\renewcommand{\subsection}{\@startsection{subsection}{2}{0mm}%
                                {-1explus -.5ex minus -.2ex}%
                                {0.5ex plus .2ex}%
                                {\normalfont\small\bfseries}}
\renewcommand{\subsubsection}{\@startsection{subsubsection}{3}{0mm}%
                                {-1ex plus -.5ex minus -.2ex}%
                                {1ex plus .2ex}%
                                {\normalfont\footnotessize\bfseries}}
\renewcommand\small{\@setfontsize\small{10}{11}}                           
\makeatother

% Define BibTeX command
\def\BibTeX{{\rm B\kern-.05em{\sc i\kern-.025em b}\kern-.08em
    T\kern-.1667em\lower.7ex\hbox{E}\kern-.125emX}}

% Don't print section numbers
\setcounter{secnumdepth}{0}

\setlength{\parindent}{0pt}
\setlength{\parskip}{1pt plus 0.5ex}

\newcommand{\tab}{\hspace{.02\textwidth}}
\newcommand{\ds}{\displaystyle}
\newcommand{\defn}[1]{\textbf{Def} (\emph{#1})}
\newcommand{\thm}[1]{\textbf{Thm} (\emph{#1})}
\newcommand{\set}[1]{\left\{ #1 \right\}}
\newcommand{\gf}[1]{\text{GF}(#1)}

\renewcommand{\dv}[2]{\frac{d#1}{d#2}}

% Redefine some commands for newtxmath boldness
\renewcommand{\grad}{\nabla}
\renewcommand{\curl}[1]{\nabla\times#1}
\renewcommand{\div}[1]{\nabla\cdot#1}
\renewcommand{\cross}{\times}

% -----------------------------------------------------------------------

\begin{document}

\raggedright
\footnotesize
\begin{multicols*}{3}

% multicol parameters
% These lengths are set only within the two main columns
%\setlength{\columnseprule}{0.25pt}
\setlength{\premulticols}{1pt}
\setlength{\postmulticols}{1pt}
\setlength{\multicolsep}{1pt}
\setlength{\columnsep}{2pt}

\begin{center}
	\Large{\underline{ELEC 433 Formula Sheet}}
\end{center}

\section{Coding Approaches and Characteristics}

Channel Capacity for Additive White Gaussian for bandwidth $W$ and noise $N_0$:\\
\tab $\ds C = W\log_2\left(1 + \frac{P}{N_0 W}\right)$

BFSK bit error probability:\\
\tab $\ds p = \frac{1}{2}e^{-E_b / 2N_0}$

\section{Binary Linear Block Codes}
$(n,k,d)$ code:\\
\begin{itemize}[itemsep=0em]
	\item $n$ - length of codeword
	\item $k$ - number of message bits in codeword
	\item $d$ - Code minimum distance
\end{itemize}


Number of codewords in a code C:\\
\tab $\abs{C} = M = 2^k$

Code rate:\\
\tab $\ds R = \frac{\log_2(M)}{n} = \frac{k}{n}$

Vector space dimensions:\\
\tab $\dim{S} + \dim{S^\perp} = \dim{V}$

\defn{Binary Linear Block Codes}: A subset $C \subseteq V_n$ is a binary linear block code if:
\begin{itemize}[itemsep=0em]
    \itemsep 0em
    \item $\vb{u} + \vb{v} \in C \quad \forall \vb{u}, \vb{v} \in C$
    \item $a\vb{u} \in C \quad \forall \vb{u} \in C, a \in \set{0,1}$
\end{itemize}

Hamming Weight:\\
\tab $w(\vb{x}) =$ number of non-zero elements in $\vb{x}$

Hamming Distance:\\
\tab $d(\vb{x}, \vb{y}) =$ number of places in which $\vb{x}$ and $\vb{y}$ differ

Hamming Distance for binary linear codes:\\
\tab $d(\vb{x}, \vb{y}) = w(\vb{x} + \vb{y})$

Minimum Hamming Distance:\\
\vspace{-0.5em}
\begin{itemize}[itemsep=0em]
    \item $d(C) = \min{\set{d(\vb{x}, \vb{y}) : \vb{x}, \vb{y} \in C, \vb{x} \neq \vb{y}}}$
    \item A code C can detect up to $v$ errors if $d(C) \geq v + 1$
    \item A code C can correct up to $t$ errors if $d(C) \geq 2t + 1$
\end{itemize}

Singleton Bound:\\
\tab $d_\text{min} \leq n - k + 1$

\defn{Generator Matrix}: A $k\times n$ matrix whose rows for a basis for a linear $(n,k)$ code of a subspace $C$ is said to be a generator matrix for $C$.

\section{Groups, Rings, and Fields}

\defn{Group} A group $(G, \cdot)$ is a set of objects $G$ on which a binary operation $\cdot$ is defined: $a \cdot b \in G : \forall a,b \in G$. The operation must satisfy:
\vspace{-0.5em}
\begin{itemize}[itemsep=0em]
    \item Associativity: $a \cdot (b \cdot c) = (a \cdot b) \cdot c$
    \item Identity: $\exists e \in G \mid \forall a \in G, a \cdot e = a$
    \item Inverse: $\forall a \in G, \exists \text{ unique } a^{-1} \in G \mid a \cdot a^{-1} = e$
\end{itemize}

\defn{Commutative Group} A group is said to be commutative or abelian if it also satisfies: $\forall a,b \in G, a \cdot b = b \cdot a$

\defn{Ring} A ring $(R, +, \cdot)$ is a set of objects $R$ on which two binary operations ($+$ and $\cdot$) are defined. It has properties:
\vspace{-0.5em}
\begin{itemize}[itemsep=0em]
    \item $(R, +)$ is a commutative group under $+$ with identity "$0$"
    \item Associativity: $a \cdot (b \cdot c) = (a \cdot b) \cdot c$
    \item Distribution: $a \cdot (b + c) = (a \cdot b) + (a \cdot c)$
\end{itemize}

\defn{Commutative Ring} A ring is said to be commutative if it also satisfies: $\forall a,b \in G, a \cdot b = b \cdot a$

\defn{Ring with Identity} A ring is said to be a ring with identity if the operation $\cdot$ has an identity element "$1$"

\defn{Division Ring} Let $(R, +, \cdot)$ be a ring, and $R^* = R - {0}$. If the ring is a commutative ring with identity, and $(R^*, \cdot)$ is a group, then the ring is said to be a division ring.

\defn{Field} A field $(F, +, \cdot)$ is a set of objects $F$ for which two binary operations ($+$ and $\cdot$) are defined. $F$ is said to be a field if and only if:
\vspace{-0.5em}
\begin{itemize}[itemsep=0em]
    \item $(F,+)$ is a commutative group under $+$ with additive identity "$0$"
    \item $(F^*, \cdot)$ is a commutative group under $\cdot$ with multiplicative identity "$1$"
    \item Distribution: $a \cdot (b + c) = (a \cdot b) + (a \cdot c)$
\end{itemize}

Finite Integer Fields:\\
\tab $S = \set{0, 1, \ldots, p - 1}$ form a finite field if $p$ is prime.

Properties of Finite Fields:\\
\vspace{-0.5em}
\begin{itemize}[itemsep=0em]
    \item Order: A field of order $q$ has cardinality $\abs{F} = q$, denoted $\gf{q}$
    \item Let $\beta \in \gf{q}, \beta \neq 0$. The order of $\beta$ is the smallest positive integer $m$ such that $\beta^m = 1$
    \item If $t$ is the order of $\beta$, then $t \mid (q - 1)$
    \item In any finite field, there are on or more elements of order $q-1$ called primitive elements.
\end{itemize}

Euler's Totient Function:\\
\tab $\phi(t) =$ number of positive integers less than $t$ that are relatively prime to $t$

Finite Fields and Euler's Totient Function:\\
\vspace{-0.5em}
\begin{itemize}[itemsep=0em]
    \item The number of elements in $\gf{q}$ of order $t$ is $\phi(t)$
    \item In $\gf{q}$ there are exactly $\phi(q - 1)$ primitive elements
    \item If $\alpha$ is a primitive element, then $1, \alpha, \alpha^2, \ldots, \alpha^{q-2}$ must be non-zero elements of $\gf{q}$
\end{itemize}

\defn{Primitive Polynomial} If an irriducible polynomial $p(x)$ such that the smallest positive integer $n$ for which $p(x)$ divides $x^n - 1$ is $n=p^m - 1$ for a prime $p$ and positive integer $m$, the polynomial is said to be a primitive polynomial.

\section{Encoding and Decoding}
Codeword convention: Data appears unaltered at the start of the code word.

\thm{Equivalence of Binary Linear Codes} Two linear binary codes are called equivalent if one can be obtained from the other by permuting the positions of the code. Two $k\times n$ mbinary matrices generate equivalent linear $(n,k,d)$ codes if one matrix can be obtained from the otehr by a sequence of row, column permutations and row addition.

\thm{Systematic Codes} Let $G$ be a generator matrix of an $(n,k)$ code. Then $G$ can be transformed to the form $\begin{bmatrix}
	I_k \mid P
\end{bmatrix}$ where $P$ is called the parity matrix.

Encoding of a message $\vb{m}$ with a code $C$:\\
\tab $c = \vb{m}G$

\defn{Parity Check Matrix}. $H$ satisfies $GH^T = 0$ and is a basis for the dual space. In systematic form $H = \begin{bmatrix}
	P^T \mid I_{n-k}
\end{bmatrix}$

Syndrome of a received word $\vb{r}$:\\
\tab $\vb{s} = \vb{r}H^T$

\section{Hamming Codes}
\defn{Binary Hamming Code} Let $m \in \mathbb{Z}$ and $H$ be a $m \times (2^m-1)$ matrix with columns which are the non-zero distinct words from a vector space $V_m$. The code having $H$ as its parity-check matrix is a binary Hamming code of length $2^m-1$

Hamming code parameters:\\
\tab $C: (2^m-1, 2^m-1-m, 3) \quad C^\perp: (2^m-1, m, 2^{m-1})$

Decoding Hamming Codes where columns of $H$ are arranged in order of increasing binary numbers:
\vspace{-0.5em}
\begin{enumerate}[itemsep=0em]
	\item Compute $S(\vb{r}) = \vb{r}H^T$
	\item If $S(\vb{r}) = 0$, then $\vb{r}$ is a valid codeword
	\item Else, $S(\vb{r})$ gives the binary position of the error 
\end{enumerate}

Hamming Bound:\\
\tab $\ds \sum_{i=0}^{t} {n \choose i} \leq 2^{n-k}$

\section{Cyclic Codes}
\defn{Cyclic Code} A code $C$ is cyclic if $C$ is linear and a cyclic shift of any codeword is another codeword.

Properties of a $(n,k)$ binary cyclic code $C$:\\
\vspace{-0.5em}
\begin{enumerate}[itemsep=0em]
	\item There exists a generator polynomial of minimal degree $n-k$
	\item Every code polynomial in $C$ can be expressed as $c(x) = m(x)g(x)$ where $m(x)$ has degree $< k-1$
	\item We can write $x^n-1 = g(x)h(x)$ where $h(x)$ is the parity check polynomial.
	\item If $g(x)$ is a primitive polynomial then $C$ is also a Hamming code.
\end{enumerate}

Generator Matrix:\\
\tab $G = \begin{bmatrix}
	g_0 & g_1 & \cdots & g_{n-k} & 0 & \cdots & 0\\
	0 & g_0 & g_1 & \cdots & g_{n-k} & 0 & 0\\
	0 & 0 & \ddots & \ddots & & \ddots & 0\\
	0 & 0 & 0 & g_0 & g_1 & \cdots & g_{n-k}
\end{bmatrix}$

Parity Check Matrix:\\
\tab $h^*(x) = x^kh(x^{-1}) = h_k + h_{k-1}x + \ldots + h_0x^k$\\
\tab $H = \begin{bmatrix}
	h_k & \cdots & h_1 & h_0 & 0 & \cdots & 0\\
	0 & h_k & \cdots & h_1 & h_0 & 0 & 0\\
	0 & 0 & \ddots & \ddots & & \ddots & 0\\
	0 & 0 & 0 & h_k & \cdots & h_1 & h_0
\end{bmatrix}$

\columnbreak
Systematic Generator Matrix:\\
\vspace{-0.5em}
\begin{enumerate}[itemsep=0em]
	\setlength\itemsep{0em}
	\item For $i = n-k$ to $n-1$, compute $x^i \mod g(x) = p_i(x)$
	\item Rows of $G$ are formed by $x^i + p_i(x)$.
\end{enumerate}

Systematic Encoding:\\
\tab $c(x) = m(x)x^{n-k} + m(x)x^{n-k}\mod g(x)$

\defn{Shortened Cyclic Code} $(n, k) \rightarrow (n-u, k-u)$ by setting $u$ most significant bits of codeword to zero. Typically not cyclic. A CRC code is a shortened cyclic code.

\section{Minimal Polynomials}
\defn{Minimal Polynomial} Let $\alpha \in \gf{2^m}$. $p(x)$ is a minimal polynomial of $\alpha$ with respect to $\gf{2^m}$ if it is the smallest degree monic polynomial such that $p(\alpha) = 0$.

Properties of Minimal Polynomials:\\
\vspace{-0.5em}
\begin{itemize}[itemsep=0em]
	\item The degree of $p(x)$ is $d$, and $d \mid m$
	\item $f(\alpha) = 0 \implies p(x) \mid f(x)$
	\item $p(x)$ is irreducible in $\gf{2^m}$
	\item If $\alpha$ is primitive, $p(x)$ is a primitive polynomial.
\end{itemize}

\defn{Conjugacy Class} Let $\alpha \in \gf{2^m}$. The conjugacy class of $\alpha$ is $\set{\alpha, \alpha^2, \alpha^{2^2}, \cdots, \alpha^{2^{d-1}}}$. If $p(\alpha) = 0$, any element of the conjugacy class is also a root.

\defn{Cyclotomic Coset} The partition of powers of $\alpha$ by the conjugacy classes of a finite field is called the set of cyclotomic cosets.

\section{BCH Codes}
BCH codes are a generalization of cyclic Hamming codes. The generator polynomial $g(x)$ is a primitive polynomial. Codeword satisfies $g(\alpha) = 0 \implies c(\alpha) = 0$.

\thm{BCH Bound} Let $C$ be a $(n,k)$ 2-ary cyclic code with generator polynomial $g(x)$. Let $\alpha \in \gf{2^m}$ be an element of order $n$, $n \mid 2^m - 1$. If $g(x)$ is a minimal polynomial with roots $\alpha^b, \alpha^{b+1}, \cdots, \alpha^{b+\delta-2}$, then $C$ has minimum distance at least $\delta$. $g(x)$ is degree $n-k$ and is the product of the minimal polynomials of the roots\\
\tab $g(x) = \text{LCM}\{m_b(x), m_{b+1}(x), \cdots, m_{b+\delta-2}(x)\}$

\defn{Narrow-Sense BCH Code} Narrow-Sense codes have parameter $b = 1$.

\defn{Binary Primitive BCH Codes} For any $m$ and $t < n/2$, there exists a binary primitive BCH code with parameters $n = 2^m - 1$, $d \geq 2t = 1$, $n-k \leq mt$, where $d$ is the designed distance.

Construction of a $t$ error correcting 2-ary BCH Code:\\
\vspace{-0.5em}
\begin{enumerate}[itemsep=0em]
	\item Find $\alpha \in \gf{2^m}$ where $m$ is minimal.
	\item Select $2t$ consecutive powers of $\alpha$ starting at $\alpha^b$.
	\item Find $g(x)$ as the LCM of the minimal polynomials for those powers of $\alpha$.
\end{enumerate}

\section{Reed-Solomon Codes}
Reed-Solomon codes are a subset of BCH codes and are non-binary. Properties:
\vspace{-0.5em}
\begin{itemize}[itemsep=0em]
	\item $\alpha$ is primitive.
	\item Generator polynomial $g(x) = (x-\alpha)(x-\alpha^2)\cdots(x-\alpha^{2t})$
	\item $n = 2^m -1 \qquad n - k = 2t$
\end{itemize}

A RS code can correct up to $s$ errors and $r$ erasures if\\
\tab $2s + r < 2t$

\section{Convolutional Codes}
\defn{Constraint Length} The constraint length $L$ is the length of longest input shift register with maximum number of memory elements plus one.

Coding Rate:\\
\tab $\ds R = \frac{\text{symbols shifted in a cycle}}{\text{number of output symbols}}$

\section{LPDC Codes}
\subsection{Parity Check Matrix Representation}
\begin{itemize}[itemsep=0em]
	\item Let $W_r$ be the number of $1$'s in each row
	\item Let $W_c$ be the number of $1$'s in each column
	\item A matrix is called low density if $W_c \ll n$ and $W_r \ll (n-k)$
\end{itemize}

\subsection{Graph Representation}
\begin{itemize}[itemsep=0em]
	\item Node are separated into variable nodes $f_i$ and check nodes $c_j$
	\item An edge connects nodes $f_i$ and $c_j$ if $H_{ij}$ = 1
\end{itemize}

\defn{Regular LPDC Code} A LPDC code is said to be regular if $W_c$ and $W_r = W_c(n/(n-l))$ are constant.

% Footer content
\rule{0.3\linewidth}{0.25pt}
\scriptsize\\
Updated \today\\
\href{https://github.com/DonneyF/formula-sheets}{https://github.com/DonneyF/formula-sheets}
\end{multicols*}
\end{document}
