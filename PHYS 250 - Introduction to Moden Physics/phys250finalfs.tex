\documentclass[12pt,landscape]{article}
\usepackage{multicol}
\usepackage{calc}
\usepackage{ifthen}
\usepackage[landscape]{geometry}
\usepackage{amsmath,amsthm,amsfonts,amssymb}
\usepackage{color,graphicx,overpic}
\usepackage{hyperref}
\usepackage{enumitem}
\usepackage{physics}
\usepackage{newtxtext,newtxmath}

% This sets page margins to .5 inch if using letter paper, and to 1cm
% if using A4 paper. (This probably isn't strictly necessary.)
% If using another size paper, use default 1cm margins.
\ifthenelse{\lengthtest { \paperwidth = 11in}}
{ \geometry{top=.5in,left=.5in,right=.5in,bottom=.5in} }
{\ifthenelse{ \lengthtest{ \paperwidth = 297mm}}
	{\geometry{top=1cm,left=1cm,right=1cm,bottom=1cm} }
	{\geometry{top=1cm,left=1cm,right=1cm,bottom=1cm} }
}

% Turn off header and footer
\pagestyle{empty}

% Redefine section commands to use less space
\makeatletter
\renewcommand{\section}{\@startsection{section}{1}{0mm}%
	{-1ex plus -.5ex minus -.2ex}%
	{0.5ex plus .2ex}%x
	{\normalfont\large\bfseries}}
\renewcommand{\subsubsection}{\@startsection{subsection}{2}{0mm}%
	{-1explus -.5ex minus -.2ex}%
	{0.5ex plus .2ex}%
	{\normalfont\normalsize\bfseries}}
\renewcommand{\subsubsection}{\@startsection{subsubsection}{3}{0mm}%
	{-1ex plus -.5ex minus -.2ex}%
	{1ex plus .2ex}%
	{\normalfont\small\bfseries}}
\makeatother

% Define BibTeX command
\def\BibTeX{{\rm B\kern-.05em{\sc i\kern-.025em b}\kern-.08em
		T\kern-.1667em\lower.7ex\hbox{E}\kern-.125emX}}

% Don't print section numbers
\setcounter{secnumdepth}{0}

\setlength{\parindent}{0pt}
\setlength{\parskip}{1pt plus 0.5ex}

\newcommand{\tab}{\hspace{.02\textwidth}}
\newcommand{\ds}{\displaystyle}

% -----------------------------------------------------------------------

\begin{document}
\raggedright
\footnotesize
\begin{multicols}{3}
	
	
	% multicol parameters
	% These lengths are set only within the two main columns
	%\setlength{\columnseprule}{0.25pt}
	\setlength{\premulticols}{1pt}
	\setlength{\postmulticols}{1pt}
	\setlength{\multicolsep}{1pt}
	\setlength{\columnsep}{2pt}
	
	\begin{center}
		\Large{\underline{PHYS 250 Final Formula Sheet}}
	\end{center}
	
	\section{Energy \& Waves}
	\tab $E_{\text{max}}^2 \approx (\text{amplitude})^2$\\
		$\bullet$ Does not depend on frequency or matter.
	
	Energy of a photon:\\
	\tab $\ds E = hf = \frac{hc}{\lambda} $\\
		$\bullet$ Interference is evidence that light is a wave.
		
	\section{Photoelectric Effect}
		\begin{itemize}[leftmargin=0.5cm]
			\itemsep0em
			\item The value of $f_0$ depends on cathode material.
			\item $V_{\text{stop}}$ is independent of light intensity.
			\item Number of electrons $\propto$ Intensity
			\item Maximum $E_k \propto$ Frequency
		\end{itemize}
		Stopping Potential: \tab $\ds V_{\text{stop}} = \frac{hf - E_0}{e}$\\
		The Photon Rate: \tab $\ds P = \dv{N}{t} hf$
	
	\section{Emission and Absorption}
	\begin{itemize}[leftmargin=0.5cm]
		\itemsep0em
		\item Atom transitions to higher energy state by absorbing a photon. Emits a photon of the same frequency if it jumps back.
		\item Stimulated Emission: Production of two identical photons by one photon interacting with an excited atom. Only occurs if the first photon's frequency matches the energy difference.
		\item Population Inversion: Having proportionally larger excited atoms than the number of non-excited atoms.
	\end{itemize}
	
	Balmer's Formula ($\lambda$ in hydrogen spectrum):\\
	\tab $\ds \frac{91.18\text{nm}}{\frac{1}{m^2} - \frac{1}{n^2}} \text{ for m = 1,2,3... \& } n > m$
	
\section{Bohr Model}
	\begin{itemize}[leftmargin=0.5cm]
		\itemsep0em
		\item Electrons exist only in certain orbits. A particular arrangement of electrons is a stationary state.
		\item Each stationary state has a discrete energy.
	\end{itemize}

	Hydrogen radius: \tab $r_n = n^2 a_B$
	
	Hydrogen Energy: \tab $E_n = -13.60 \text{ eV} / n^2$

\section{Schroedinger Equation}
	$$\dv[2]{\psi}{x}+\frac{2m}{\hbar^2}[E - U(x)]\psi (x) = 0$$
	$$\hbar = h/2\pi$$
	de Broglie wavelength:\\
	\tab $ \ds \lambda = \frac{h}{p} = \frac{h}{mv} = \frac{h}{\sqrt{2m E_k}}$

\section{Potential Wells}
	\begin{itemize}[leftmargin=0.5cm]
		\itemsep0em
		\item A particle with energy $E > U_0$ an escape into the classically forbidden region.
		\item Node spacing is smaller when $E_K$ is larger
		\item Classical particle is more likely to be found where it is moving slowly
		\item $\psi (x)$ amplitude is larger where $E_K$ is smaller
	\end{itemize}
	Wave Function in the classically forbidden region:\\
	\tab $\psi (x) = \psi_{\text{edge}} e^{-(x-L)/\eta}$
	
	Penetration distance:\\
	\tab $\ds \eta = \frac{\hbar}{\sqrt{2m(U_0 - E)}}$
	\begin{itemize}[leftmargin=0.5cm]
		\itemsep0em
		\item Quantum tunneling requires no energy and has oscillatory solutions on the other side
		\item $U_0 - E$ can be the metal's work function
	\end{itemize}

	Infinite well energy:\\
	\tab $\ds E_n =\frac{n^2 \pi^2 \hbar^2}{2mL^2}$

	Tunneling Probability:\\
	\tab $P_{\text{tunnel}} = e^{-2w/\eta}$ for potential well width of $w$
	
	
\section{Measurement}
\begin{itemize}[leftmargin=0.5cm]
	\itemsep0em
	\item Measuring collapses wave function to a specific eigenstate
	\item Cannot know both position and energy.
	\item Measuring position $\rightarrow$ $\abs{\psi(x)}^2$ changes in time
	\item Measuring energy $\rightarrow$ $\abs{\psi(x)}^2$ no change in time
\end{itemize}
	
\section{Wave Packets}
	\begin{itemize}[leftmargin=0.5cm]
		\itemsep0em
		\item A localized particle with constant speed
		\item For any wave packet $\Delta f \Delta t \geq 1$
	\end{itemize}
	Uncertainty: \tab $\Delta x = v_x \Delta t = \frac{p_x}{m}\Delta t$\\
	Uncertainty Principle: \tab $\Delta x \Delta p_x \geq h / 2$
\section{Hydrogen Atom}
	Bohr Radius: \tab $\ds a_B = \frac{4 \pi \varepsilon_0 \hbar^2}{me^2}$
	
	Energy:\\
	\tab $E_n = -13.60 eV / n^2$, $n = 1, 2, 3...$
	
	Momentum:\\
	\tab $L = \sqrt{l(l+1)}\hbar$, $l = 0,1,2...n-1$\\
	\tab $L_z = m\hbar$, $m = -l, -l+1,...0,...l-1,l$
	
	Symbols for $l$:\\
	\tab 
	$0 \rightarrow s$,
	$1 \rightarrow p$,
	$2 \rightarrow d$,
	$3 \rightarrow f$

	Radial probability: \tab $P_r(r) = 4\pi r^2 \abs{R_{nl}(r)}$
	
	Spin: \tab $S_z = m_s \hbar$, $m_s = \pm 1/2$
	
	Spin Angular Momentum: \tab $S=\sqrt{3}/2\hbar$
	
	Pauli Exclusion Principle: No two electrons can have the same set of quantum numbers. If one electron is present in a state, it excludes all others.
	
	High l $\rightarrow$ circular orbit
	
	\section{Special Relativity}
	\begin{itemize}[leftmargin=0.5cm]
		\itemsep0em
		\item Laws of physics are the same in all inertial frames
		\item Any two events occurring simultaneously in one reference frame are not simultaneous in any reference frame moving relative to the original.
		\item Proper time: The time interval between two events occurring in the same position.
	\end{itemize}
	\tab $\ds \gamma = \frac{1}{\sqrt{1-(\frac{v}{c})^2}}$ \hspace{0.5cm} $\ds \gamma_p = \frac{1}{\sqrt{1-(\frac{u}{c})^2}}$
	
	Time Dilation:
	\tab $\Delta t = \gamma \Delta \tau$
	
	Length Contraction:
	\tab $L' = \frac{L}{\gamma}$
	
	Spacetime Interval:
	\tab $s^2 = c^2(\Delta t)^2 - (\Delta x)^2$
	
	Relativistic Momentum:
	\tab $p = \gamma_p m u$
	
	Relativistic Energy:\\
	\tab $E = \gamma_p mc^2 = E_0 + K = mc^2 + (\gamma_p - 1)mc^2$\\
	\tab $pc = \frac{u}{c}E$
	
	% Footer content
	\rule{0.3\linewidth}{0.25pt}
	\scriptsize\\
	Updated \today\\
	\href{https://github.com/DonneyF/formula-sheets}{https://github.com/DonneyF/formula-sheets}
\end{multicols}
\end{document}