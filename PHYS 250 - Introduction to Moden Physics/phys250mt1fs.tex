\documentclass[12pt,landscape]{article}
\usepackage{multicol}
\usepackage{calc}
\usepackage{ifthen}
\usepackage[landscape]{geometry}
\usepackage{amsmath,amsthm,amsfonts,amssymb}
\usepackage{color,graphicx,overpic}
\usepackage{hyperref}
\usepackage{enumitem}
\usepackage{physics}
\usepackage{newtxtext,newtxmath}

% This sets page margins to .5 inch if using letter paper, and to 1cm
% if using A4 paper. (This probably isn't strictly necessary.)
% If using another size paper, use default 1cm margins.
\ifthenelse{\lengthtest { \paperwidth = 11in}}
{ \geometry{top=.5in,left=.5in,right=.5in,bottom=.5in} }
{\ifthenelse{ \lengthtest{ \paperwidth = 297mm}}
	{\geometry{top=1cm,left=1cm,right=1cm,bottom=1cm} }
	{\geometry{top=1cm,left=1cm,right=1cm,bottom=1cm} }
}

% Turn off header and footer
\pagestyle{empty}

% Redefine section commands to use less space
\makeatletter
\renewcommand{\section}{\@startsection{section}{1}{0mm}%
	{-1ex plus -.5ex minus -.2ex}%
	{0.5ex plus .2ex}%x
	{\normalfont\large\bfseries}}
\renewcommand{\subsubsection}{\@startsection{subsection}{2}{0mm}%
	{-1explus -.5ex minus -.2ex}%
	{0.5ex plus .2ex}%
	{\normalfont\normalsize\bfseries}}
\renewcommand{\subsubsection}{\@startsection{subsubsection}{3}{0mm}%
	{-1ex plus -.5ex minus -.2ex}%
	{1ex plus .2ex}%
	{\normalfont\small\bfseries}}
\makeatother

% Define BibTeX command
\def\BibTeX{{\rm B\kern-.05em{\sc i\kern-.025em b}\kern-.08em
		T\kern-.1667em\lower.7ex\hbox{E}\kern-.125emX}}

% Don't print section numbers
\setcounter{secnumdepth}{0}

\setlength{\parindent}{0pt}
\setlength{\parskip}{1pt plus 0.5ex}

\newcommand{\tab}{\hspace{.02\textwidth}}
\newcommand{\ds}{\displaystyle}

% -----------------------------------------------------------------------

\begin{document}
\raggedright
\footnotesize
\begin{multicols}{3}
	
	
	% multicol parameters
	% These lengths are set only within the two main columns
	%\setlength{\columnseprule}{0.25pt}
	\setlength{\premulticols}{1pt}
	\setlength{\postmulticols}{1pt}
	\setlength{\multicolsep}{1pt}
	\setlength{\columnsep}{2pt}
	
	\begin{center}
		\Large{\underline{PHYS 250 MT1 Formula Sheet}}
	\end{center}
	
	\section{Energy \& Waves}
	\tab $E_{\text{max}}^2 \approx (\text{amplitude})^2$\\
		$\bullet$ Does not depend on frequency or matter.
	
	Energy of a photon:\\
	\tab $\ds E = hf = \frac{hc}{\lambda} $\\
		$\bullet$ Interference is evidence that light is a wave.
	
	\section{Photoelectric Effect}
	\begin{itemize}[leftmargin=0.5cm]
		\itemsep0em
		\item Current is linearly proportional to intensity.
		\item Current appears without delay.
		\item Photoelectrons are only emitted if the light frequency $f$ exceeds a threshold frequency $f_0$
		\item The value of $f_0$ depends on cathode material.
		\item Current becomes independent of $V$ for large $V$.
		\item If the voltage is made negative, then the current decreases until some stopping potential.
		\item $V_{\text{stop}}$ is independent of light intensity.
		\item Electron immediately has enough energy to escape.
		\item Number of electrons $\propto$ Intensity
		\item Maximum $E_k \propto$ Frequency
	\end{itemize}

	Classical Interpretation:
	\begin{itemize}[leftmargin=0.5cm]
		\itemsep0em
		\item Metal is heated to high temperature to allow Thermal Emission
		\item Can possibly raise electron temperature to significantly higher than the metal, so that the electron can emit without the metal melting.
		\item Nothing to suggest threshold frequency
		\item Does not explain instantaneous current
		\item Does not explain why $V_{\text{stop}}$ is constant.
	\end{itemize}

	Stopping Potential:\\
	\tab $\ds V_{\text{stop}} = \frac{hf - E_0}{\text{e}}$
	
	The Photon Rate:\\
	\tab $\ds P = \dv{N}{t} hf$
	
	\subsection{Emission and Absorption}
	\begin{itemize}[leftmargin=0.5cm]
		\itemsep0em
		\item Atom jumps from lower energy to higher energy state by absorbing a photon. It can emit a photon of the same frequency as it jumps back. (Spontaneous Transmission)
		\item Stimulated Emission: Production of two identical photons by one photon interacting with an excited atom. Only occurs if the first photon's frequency matches the energy difference.
		\item A laser uses a chain reaction of stimulated emission in many excited atoms. The number of excited atoms must out number the non-excited atoms to be stable.
		\item Population Inversion: Having an amount of atoms N such that the number of excited atoms is proportionally larger than the number of non-excited atoms.
	\end{itemize}
	
	Balmer's Formula ($\lambda$ in hydrogen spectrum):\\
	\tab $\ds \frac{91.18\text{nm}}{\frac{1}{m^2} - \frac{1}{n^2}} \text{ for m = 1,2,3... \& } n > m$
	
	\subsection{Wave Function}
	We want to relate the probability functions with electrons, but there are no waves for electrons. We assume there is some continuous, wave-like function for matter that is analogous for light.
	
	Probability Density:\\
	\tab $P(x) = \abs{\psi (x)}^2$
	
	Normalization:\\
	\tab $\ds \int_{-\infty}^{\infty} \abs{\psi (x)}^2 dx = 1$
	
	\subsection{Quantum Models}
	Bohr Model can't explain
	\begin{itemize}[leftmargin=0.5cm]
		\itemsep0em
		\item Why angular momentum is quantized
		\item Why electrons don't radiate energy when in orbits
		\item How does electron know what orbit to jump to?
		\item Can't be generalized
		\item Shapes of molecular orbits
		\item Molecular bonds
		\item Very closely spaced spectral lines
	\end{itemize}

	\subsection{Schroedinger Equation}
	$$\dv[2]{\psi}{x}+\frac{2m}{\hbar^2}[E - U(x)]\psi (x) = 0$$
	$$\hbar = h/2\pi$$
	de Broglie wavelength:\\
	\tab $ \ds \lambda = \frac{h}{p} = \frac{h}{mv} = \frac{h}{\sqrt{2m E_k}}$
	
	Restrictions
	\begin{itemize}[leftmargin=0.5cm]
		\itemsep0em
		\item $\psi (x)$ is continuous
		\item $\psi (x)$ = 0 if $x$ is in a region where te particle is impossible to be in
		\item $\psi (x) \rightarrow 0$ as $x \rightarrow \infty$
		\item $\psi (x)$ is a normalized function
	\end{itemize}

	\subsection{Potential Wells}
	\begin{itemize}[leftmargin=0.5cm]
		\itemsep0em
		\item A particle with energy $E > U_0$ an escape into the classically forbidden region.
		\item Particle's energy is quantized
		\item There area finite number of bound states
		\item $\psi (x)$ extends into the classically forbidden region
		\item Node spacing is smaller when kinetic energy is larger
		\item Classical particle is more likely to be found where it is moving slowly
		\item Wave function amplitude is larger where the kinetic energy is smaller
	\end{itemize}
	Wave Function in the classically forbidden region:\\
	\tab $\psi (x) = \psi_{\text{edge}} e^{-(x-L)/\eta}$
	
	Penetration distance:\\
	\tab $\ds \eta = \frac{\hbar}{\sqrt{2m(U_0 - E)}}$
	
	\begin{itemize}[leftmargin=0.5cm]
		\itemsep0em
		\item Quantum tunneling requires no energy
		\item Tunneling requires oscillatory solutions on the other side
	\end{itemize}

	Tunneling Probability:\\
	\tab $P_{\text{tunnel}} = e^{-2w/\eta}$ for some edge $x=w$
	
	% Footer content
	\rule{0.3\linewidth}{0.25pt}
	\scriptsize\\
	Updated \today\\
	\href{https://github.com/DonneyF/formula-sheets}{https://github.com/DonneyF/formula-sheets}
\end{multicols}
\end{document}