\documentclass[12pt,landscape]{article}
\usepackage{multicol}
\usepackage{calc}
\usepackage{ifthen}
\usepackage[landscape]{geometry}
\usepackage{amsmath,amsthm,amsfonts,amssymb}
\usepackage{color,graphicx,overpic}
\usepackage{hyperref}
\usepackage{enumitem}
\usepackage{physics}
\usepackage{newtxtext,newtxmath}

% This sets page margins to .5 inch if using letter paper, and to 1cm
% if using A4 paper. (This probably isn't strictly necessary.)
% If using another size paper, use default 1cm margins.
\ifthenelse{\lengthtest { \paperwidth = 11in}}
{ \geometry{top=.5in,left=.5in,right=.5in,bottom=.5in} }
{\ifthenelse{ \lengthtest{ \paperwidth = 297mm}}
	{\geometry{top=1cm,left=1cm,right=1cm,bottom=1cm} }
	{\geometry{top=1cm,left=1cm,right=1cm,bottom=1cm} }
}

% Turn off header and footer
\pagestyle{empty}

% Redefine section commands to use less space
\makeatletter
\renewcommand{\section}{\@startsection{section}{1}{0mm}%
	{-1ex plus -.5ex minus -.2ex}%
	{0.5ex plus .2ex}%x
	{\normalfont\large\bfseries}}
\renewcommand{\subsubsection}{\@startsection{subsection}{2}{0mm}%
	{-1explus -.5ex minus -.2ex}%
	{0.5ex plus .2ex}%
	{\normalfont\normalsize\bfseries}}
\renewcommand{\subsubsection}{\@startsection{subsubsection}{3}{0mm}%
	{-1ex plus -.5ex minus -.2ex}%
	{1ex plus .2ex}%
	{\normalfont\small\bfseries}}
\makeatother

% Define BibTeX command
\def\BibTeX{{\rm B\kern-.05em{\sc i\kern-.025em b}\kern-.08em
		T\kern-.1667em\lower.7ex\hbox{E}\kern-.125emX}}

% Don't print section numbers
\setcounter{secnumdepth}{0}

\setlength{\parindent}{0pt}
\setlength{\parskip}{1pt plus 0.5ex}

\newcommand{\tab}{\hspace{.02\textwidth}}
\newcommand{\ds}{\displaystyle}

% -----------------------------------------------------------------------

\begin{document}
\raggedright
\footnotesize
\begin{multicols}{3}
	% multicol parameters
	% These lengths are set only within the two main columns
	%\setlength{\columnseprule}{0.25pt}
	\setlength{\premulticols}{1pt}
	\setlength{\postmulticols}{1pt}
	\setlength{\multicolsep}{1pt}
	\setlength{\columnsep}{2pt}
	
	\begin{center}
		\Large{\underline{PHYS 250 MT2 Cheat Sheet}}
	\end{center}
	
	\section{Energy \& Waves}
	\tab $E_{\text{max}}^2 \approx (\text{amplitude})^2$\\
		$\bullet$ Does not depend on frequency or matter.
	
	Energy of a photon:\\
	\tab $\ds E = hf = \frac{hc}{\lambda} $\\
		$\bullet$ Interference is evidence that light is a wave.
	
	\section{Emission and Absorption}
	\begin{itemize}[leftmargin=0.5cm]
		\itemsep0em
		\item Atom jumps from lower energy to higher energy state by absorbing a photon. It can emit a photon of the same frequency as it jumps back. (Spontaneous Transmission)
		\item Stimulated Emission: Production of two identical photons by one photon interacting with an excited atom. Only occurs if the first photon's frequency matches the energy difference.
		\item A laser uses a chain reaction of stimulated emission in many excited atoms. The number of excited atoms must out number the non-excited atoms to be stable.
		\item Population Inversion: Having an amount of atoms N such that the number of excited atoms is proportionally larger than the number of non-excited atoms.
	\end{itemize}
	
	Balmer's Formula ($\lambda$ in hydrogen spectrum):\\
	\tab $\ds \frac{91.18\text{nm}}{\frac{1}{m^2} - \frac{1}{n^2}} \text{ for m = 1,2,3... \& } n > m$
	
\section{Bohr Model}
	\begin{itemize}[leftmargin=0.5cm]
		\itemsep0em
		\item Electrons can exist only in certain orbits. A particular arrangement of electrons is a stationary state.
		\item Each stationary state has a discrete energy.
	\end{itemize}

	Hydrogen radius:\\
	\tab $r_n = n^2 a_B$
	
	Hydrogen Energy:\\
	\tab $E_n = -13.60 \text{ eV} / n^2$
	
	Bohr Model can't explain
	\begin{itemize}[leftmargin=0.5cm]
		\itemsep0em
		\item Why angular momentum is quantized
		\item Why electrons don't radiate energy when in orbits
		\item How does electron know what orbit to jump to?
		\item Can't be generalized
		\item Shapes of molecular orbits
		\item Molecular bonds
		\item Very closely spaced spectral lines
	\end{itemize}

\section{Schroedinger Equation}
	$$\dv[2]{\psi}{x}+\frac{2m}{\hbar^2}[E - U(x)]\psi (x) = 0$$
	$$\hbar = h/2\pi$$
	de Broglie wavelength:\\
	\tab $ \ds \lambda = \frac{h}{p} = \frac{h}{mv} = \frac{h}{\sqrt{2m E_k}}$
	
	Restrictions
	\begin{itemize}[leftmargin=0.5cm]
		\itemsep0em
		\item $\psi (x)$ is continuous
		\item $\psi (x)$ = 0 if $x$ is in a region where te particle is impossible to be in
		\item $\psi (x) \rightarrow 0$ as $x \rightarrow \infty$
		\item $\psi (x)$ is a normalized function
	\end{itemize}

\section{Potential Wells}
	\begin{itemize}[leftmargin=0.5cm]
		\itemsep0em
		\item A particle with energy $E > U_0$ an escape into the classically forbidden region.
		\item Particle's energy is quantized
		\item There area finite number of bound states
		\item $\psi (x)$ extends into the classically forbidden region
		\item Node spacing is smaller when kinetic energy is larger
		\item Classical particle is more likely to be found where it is moving slowly
		\item Wave function amplitude is larger where the kinetic energy is smaller
	\end{itemize}
	Wave Function in the classically forbidden region:\\
	\tab $\psi (x) = \psi_{\text{edge}} e^{-(x-L)/\eta}$
	
	Penetration distance:\\
	\tab $\ds \eta = \frac{\hbar}{\sqrt{2m(U_0 - E)}}$
	
	\begin{itemize}[leftmargin=0.5cm]
		\itemsep0em
		\item Quantum tunneling requires no energy
		\item Tunneling requires oscillatory solutions on the other side
		\item $U_0 - E$ can be the metal's work function
	\end{itemize}

	Infinite well energy:\\
	\tab $\ds E_n =\frac{n^2 \pi^2 \hbar^2}{2mL^2}$

	Tunneling Probability:\\
	\tab $P_{\text{tunnel}} = e^{-2w/\eta}$ for potential well width of $w$
	
\section{Wave Packets}
	\begin{itemize}[leftmargin=0.5cm]
		\itemsep0em
		\item A localized particle
		\item Travels with constant speed
		\item For any wave packet $\Delta f \Delta t \geq 1$
	\end{itemize}

	Uncertainty:\\
	\tab $\ds \Delta x = v_x \Delta t = \frac{p_x}{m}\Delta t$
	
	Uncertainty Principle:\\
	\tab $\Delta x\, \Delta p_x \geq h / 2$
	
\section{Measurement}
	\begin{itemize}[leftmargin=0.5cm]
		\itemsep0em
		\item Measuring changes the system
		\item Measuring collapses wavefunciton to a specific eigenstate
		\item Cannot know both position and energy.
		\item Measuring position $\rightarrow$ Probability density changes with time
		\item Measuring energy $\rightarrow$ Probability density does not change
	\end{itemize}
	
\section{Hydrogen Atom}
	Bohr Radius:\\
	\tab $\ds a_B = \frac{4 \pi \varepsilon_0 \hbar^2}{me^2}$
	
	Energy:\\
	\tab $E_n = -13.60 \text{ eV} / n^2$, $n = 1, 2, 3...$
	
	Momentum:\\
	\tab $L = \sqrt{l(l+1)}\,\hbar$, $l = 0,1,2...n-1$\\
	\tab $L_z = m\hbar$, $m = -l, -l+1,...0,...l-1,l$
	
	Symbols for $l$:\\
	\tab 
	$0 \rightarrow s$,
	$1 \rightarrow p$,
	$2 \rightarrow d$,
	$3 \rightarrow f$

	% Footer content
\rule{0.3\linewidth}{0.25pt}
\scriptsize\\
Updated \today\\
\href{https://github.com/DonneyF/formula-sheets}{https://github.com/DonneyF/formula-sheets}
\end{multicols}
\end{document}