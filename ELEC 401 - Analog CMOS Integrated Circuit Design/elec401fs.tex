% !Tex program = pdflatex

\documentclass[12pt,landscape]{article}
\usepackage{multicol}
\usepackage{calc}
\usepackage{ifthen}
\usepackage[landscape]{geometry}
\usepackage{amsmath,amsthm,amsfonts,amssymb}
\usepackage{color,graphicx,overpic}
\usepackage{hyperref}
\usepackage{enumitem}
\usepackage{upgreek}
\usepackage[italicdiff]{physics}
\usepackage{newtxtext,newtxmath}
\usepackage{mdframed}
\usepackage{amsbsy}

% This sets page margins to .5 inch if using letter paper, and to 1cm
% if using A4 paper. (This probably isn't strictly necessary.)
% If using another size paper, use default 1cm margins.
\ifthenelse{\lengthtest { \paperwidth = 11in}}
	{ \geometry{top=.5in,left=.5in,right=.5in,bottom=.5in} }
	{\ifthenelse{ \lengthtest{ \paperwidth = 297mm}}
		{\geometry{top=1cm,left=1cm,right=1cm,bottom=1cm} }
		{\geometry{top=1cm,left=1cm,right=1cm,bottom=1cm} }
	}

% Turn off header and footer
\pagestyle{empty}
 

% Redefine section commands to use less space
\makeatletter
\renewcommand{\section}{\@startsection{section}{1}{0mm}%
                                {-1ex plus -.5ex minus -.2ex}%
                                {0.5ex plus .2ex}%x
                                {\normalfont\normalsize\bfseries}}
\renewcommand{\subsection}{\@startsection{subsection}{2}{0mm}%
                                {-1explus -.5ex minus -.2ex}%
                                {0.5ex plus .2ex}%
                                {\normalfont\small\bfseries}}
\renewcommand{\subsubsection}{\@startsection{subsubsection}{3}{0mm}%
                                {-1ex plus -.5ex minus -.2ex}%
                                {1ex plus .2ex}%
                                {\normalfont\footnotessize\bfseries}}
\makeatother

% Define BibTeX command
\def\BibTeX{{\rm B\kern-.05em{\sc i\kern-.025em b}\kern-.08em
    T\kern-.1667em\lower.7ex\hbox{E}\kern-.125emX}}

% Don't print section numbers
\setcounter{secnumdepth}{0}


\setlength{\parindent}{0pt}
\setlength{\parskip}{1pt plus 0.5ex}

\newcommand{\tab}{\hspace*{1em}}
\newcommand{\ds}{\displaystyle}

% Redefine some commands for newtxmath boldness
\renewcommand{\grad}{\nabla}
\renewcommand{\curl}[1]{\nabla\times#1}
\renewcommand{\div}[1]{\nabla\cdot#1}
\renewcommand{\cross}{\times}
\newcommand{\defn}[1]{\textbf{Def} (\emph{#1})}
\newcommand{\thm}[1]{\textbf{Thm} (\emph{#1})}

\newcommand{\Var}[1]{\mathrm{Var}(#1)}
\newcommand{\Cov}[1]{\mathrm{Cov}(#1)}

\mdfsetup{skipabove=2pt,skipbelow=2pt, innertopmargin=-6pt, innerbottommargin=2pt, innerleftmargin=2pt, innerrightmargin=2pt}
\theoremstyle{definition}
\newmdtheoremenv{theorem}{Theorem}

% -----------------------------------------------------------------------

\begin{document}

\raggedright
\footnotesize
\begin{multicols}{3}

\raggedcolumns

% multicol parameters
% These lengths are set only within the two main columns
%\setlength{\columnseprule}{0.25pt}
\setlength{\premulticols}{1pt}
\setlength{\postmulticols}{1pt}
\setlength{\multicolsep}{1pt}
\setlength{\columnsep}{2pt}

\begin{center}
	\Large{\underline{ELEC 401 Formula Sheet}}
\end{center}

\section{MOS Transistors}
Regions of Operation:\\
\tab NMOS:
\begin{equation*}
\begin{cases}
	V_{GS} < V_{TH}& \text{Cut-off}\\
	V_{GS} > V_{TH}, V_{DS} \ll 2(V_{GS} - V_{TH}) & \text{Deep Triode}\\
	V_{GS} > V_{TH}, V_{DS} < V_{GS} - V_{TH} & \text{Triode}\\
	V_{GS} > V_{TH}, V_{DS} > V_{GS} - V_{TH}  & \text{Saturation}
\end{cases}
\end{equation*}
\tab PMOS:
\begin{equation*}
\begin{cases}
V_{SG} < \abs{V_{TH}}& \text{Cut-off}\\
V_{SG} > \abs{V_{TH}}, V_{SD} \ll 2(V_{SG} - \abs{V_{TH}}) & \text{Deep Triode}\\
V_{SG} >\abs{V_{TH}}, V_{SD} < V_{SG} - \abs{V_{TH}} & \text{Triode}\\
V_{SG} > \abs{V_{TH}}, V_{SD} > V_{SG} - \abs{V_{TH}}  & \text{Saturation}
\end{cases}
\end{equation*}

Long Channel Current Equations:\\
\tab NMOS ($I_{DS}$):
\begin{equation*}
\begin{cases}
0 & \text{Cut-off}\\
\ds \mu_n C_\text{ox} \frac{W}{L} (V_{GS} - V_{TH}) V_{DS} & \text{Deep Triode}\\[0.5em]
\ds \mu_n C_\text{ox} \frac{W}{L} \left[(V_{GS} - V_{TH}) V_{DS} - \frac{V_{DS}^2}{2}\right] & \text{Triode}\\[1em]
\ds \frac{1}{2} \mu_n C_\text{ox} \frac{W}{L} (V_{GS} - V_{TH})^2 & \text{Saturation}
\end{cases}
\end{equation*}

\tab PMOS ($I_{SD}$):
\begin{equation*}
\begin{cases}
0 & \text{Cut-off}\\
\ds \mu_p C_\text{ox} \frac{W}{L} (V_{SG} - \abs{V_{TH}}) V_{SD} & \text{Deep Triode}\\[0.5em]
\ds \mu_p C_\text{ox} \frac{W}{L} \left[(V_{SG} - \abs{V_{TH}}) V_{SD} - \frac{V_{SD}^2}{2}\right] & \text{Triode}\\[1em]
\ds \frac{1}{2} \mu_p C_\text{ox} \frac{W}{L} (V_{SG} - \abs{V_{TH}})^2 & \text{Saturation}
\end{cases}
\end{equation*}

Transconductance (NMOS):\\
\tab $\ds g_m = \eval{\pdv{I_D}{V_{GS}}}_{V_{DS}}$\\
\tab $\ds g_m = \mu_n C_\text{ox} \frac{W}{L} (V_{GS} - V_{TH})$\\
\tab $\ds g_m = \sqrt{2\mu_n C_\text{ox} \frac{W}{L} I_D} = \frac{2I_D}{V_{GS} - V_{TH}}$

Body Effect:\\
\tab $V_\text{TH} = V_\text{TH0} + \gamma\left(\sqrt{\abs{2\Phi_F + V_{SB}}} - \sqrt{\abs{2\Phi_F}}\right)$
\tab $\ds \gamma = \frac{\sqrt{2q\varepsilon_\text{si}N_\text{sub}}}{C_\text{ox}}$

Channel Length Modulation:\\
\tab $\ds I_D = \frac{1}{2} \mu_n C_\text{ox} \frac{W}{L} (V_{GS} - V_{TH})^2 (1 + \lambda V_{DS})$

Sub-threshold Conduction:\\
\tab $I_D = I_0 e^{\frac{V_{GS}}{\zeta V_T}}$

Device Capacitances:\\
\bgroup
\def\arraystretch{1.2}%
\tab \begin{tabular}{|l|c | c |c |} 
\hline
& Cut-off & Triode & Saturation\\
\hline
$C_{GS}$ & $C_\text{ov}$ & $C_\text{ov} + \frac{C_1}{2}$ & $C_\text{ov} + \frac{2}{3}C_1$\\
\hline
$C_{GD}$ & $C_\text{ov}$ & $C_\text{ov} + \frac{C_1}{2}$ & $C_\text{ov}$\\
\hline
$C_{GB}$ & $\frac{C_1 C_2}{C_1 + C_2} \leq C_{GB} \leq C_1$ & 0 & 0\\
\hline
$C_{SB}$ & $C_5$ & $C_5 + \frac{C_2}{2}$ & $C_5 + \frac{2}{3}C_2$\\
\hline
$C_{DB}$ & $C_6$ & $C_6 + \frac{C_2}{2}$ & $C_6$\\
\hline
\end{tabular}
\egroup

Small-Signal Model:\\
\tab $\ds i_D = g_m v_{GS} + \frac{v_{DS}}{r_o} + g_{mb}v_{BS}$
\tab $\ds g_{mb} = \eta g_m = \frac{\gamma}{2\sqrt{\abs{2\Phi_F + V_{SB}}}} g_m$

% Footer content
\rule{0.3\linewidth}{0.25pt}
\scriptsize\\
Updated \today\\
\href{https://github.com/DonneyF/formula-sheets}{https://github.com/DonneyF/formula-sheets}
\end{multicols}%

\end{document}
