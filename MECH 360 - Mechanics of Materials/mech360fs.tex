\documentclass[10pt,landscape]{article}
\usepackage{multicol}
\usepackage{calc}
\usepackage{ifthen}
\usepackage[landscape]{geometry}
\usepackage{amsmath,amsthm,amsfonts,amssymb}
\usepackage{color,graphicx,overpic}
\usepackage{hyperref}
\usepackage{enumitem}
\usepackage{upgreek}
\usepackage{gensymb}
\usepackage{newtxtext,newtxmath}

% This sets page margins to .5 inch if using letter paper, and to 1cm
% if using A4 paper. (This probably isn't strictly necessary.)
% If using another size paper, use default 1cm margins.
\ifthenelse{\lengthtest { \paperwidth = 11in}}
	{ \geometry{top=.5in,left=.5in,right=.5in,bottom=.5in} }
	{\ifthenelse{ \lengthtest{ \paperwidth = 297mm}}
		{\geometry{top=1cm,left=1cm,right=1cm,bottom=1cm} }
		{\geometry{top=1cm,left=1cm,right=1cm,bottom=1cm} }
	}

% Turn off header and footer
\pagestyle{empty}
 

% Redefine section commands to use less space
\makeatletter
\renewcommand{\section}{\@startsection{section}{1}{0mm}%
                                {-1ex plus -.5ex minus -.2ex}%
                                {0.5ex plus .2ex}%x
                                {\normalfont\large\bfseries}}
\renewcommand{\subsection}{\@startsection{subsection}{2}{0mm}%
                                {-1explus -.5ex minus -.2ex}%
                                {0.5ex plus .2ex}%
                                {\normalfont\normalsize\bfseries}}
\renewcommand{\subsubsection}{\@startsection{subsubsection}{3}{0mm}%
                                {-1ex plus -.5ex minus -.2ex}%
                                {1ex plus .2ex}%
                                {\normalfont\small\bfseries}}
\makeatother

% Define BibTeX command
\def\BibTeX{{\rm B\kern-.05em{\sc i\kern-.025em b}\kern-.08em
    T\kern-.1667em\lower.7ex\hbox{E}\kern-.125emX}}

% Don't print section numbers
\setcounter{secnumdepth}{0}


\setlength{\parindent}{0pt}
\setlength{\parskip}{1pt plus 0.5ex}

\newcommand{\tab}{\hspace{.02\textwidth}}
\newcommand{\ds}{\displaystyle}


% -----------------------------------------------------------------------

\begin{document}

\raggedright
\footnotesize
\begin{multicols}{3}


% multicol parameters
% These lengths are set only within the two main columns
%\setlength{\columnseprule}{0.25pt}
\setlength{\premulticols}{1pt}
\setlength{\postmulticols}{1pt}
\setlength{\multicolsep}{1pt}
\setlength{\columnsep}{2pt}

\begin{center}
	\Large{\underline{MECH 360 Formula Sheet}}
\end{center}

\section{Stress \& Strain}
Average normal stress:\\
\tab $\ds \sigma = \frac{P}{A}$

Average shear stress:\\
\tab $\ds \tau = \frac{V}{A}$

Double shear:\\
\tab $\ds \tau = \frac{P}{2A}$

Bearing stress:\\
\tab $\ds \sigma_b = \frac{P}{A} = \frac{P}{td}$

Stresses on a 2-force member\\($\theta$ measured from vertical): \\
\tab $\ds \sigma = \frac{P}{A_\perp}\cos^2\theta \hspace{1cm} \tau = \frac{P}{A_\perp}\sin\theta\cos\theta$

Factor of safety:\\
\tab $\ds \text{Factor of safety} = \frac{\text{Ultimate Load}}{\text{Allowable Load}}$

Normal strain:\\
\tab $\ds \epsilon = \frac{\delta}{L} = \frac{d\delta}{dx}$

Local shear strain (Change of $\pi/2$):\\
\tab $\gamma = \pi/2 - \theta$

\section{Axial Load}
Hooke's Law and Modulus of Elasticity:\\
\tab $\sigma = E\epsilon$

Elastic deformation under axial loading:\\
\tab $\ds \delta = \frac{FL}{AE} = \sum_{i}\frac{F_i L_i}{A_i E_i}$

Temperature change:\\
\tab $\delta_T = L_o\alpha\Delta T \hspace{1cm}$

Poisson's Ration:\\
\vspace{1mm}
\tab $\ds \nu = -\frac{\epsilon_{\text{lat}}}{\epsilon_\text{long}}$

Shear Stress-Strain Diagrams:\\
\tab $\ds G = \frac{E}{2(1 + \nu)}$\\
\tab $\tau = G\gamma \hspace{1cm} \text{(elastic region)}$

Elastic Strain Energy:\\
\tab $\ds u = \int_{0}^{\sigma} \sigma\,d\epsilon = \frac{1}{2}\frac{\sigma^2}{E}$

\section{Torsion}
Polar Moment of Inertia:\\
\tab $J = \int r^2\,dA$\\
\tab $\ds J = \frac{\pi c^4}{2}$ \hspace{1cm} (full tube)\\
\vspace{1mm}
\tab $\ds J = \frac{\pi}{2}(c^4-a^4)$ \hspace{1cm} (hollow tube)

Shear Stress:\\
\tab $\ds \tau = \frac{T\rho}{J} \hspace{1cm } \tau_{\text{max}} = \frac{Tc}{J}$

Power:\\
\tab $P = T\omega$

Angle of Twist:\\
\tab $\ds \phi = \frac{TL}{JG} = \int_{0}^{L}\frac{T(x)}{J(x)G(x)}\,dx$

Stress Concentrations:\\
\tab $\ds \tau_{\text{max}} = K\frac{Tc}{J}$

\section{Bending}
Distributed Load Intensity at each point:\\
\tab $\ds w = \frac{dV}{dx}$

Shear at each point:\\
\tab $\ds V = \frac{dM}{dx}$

Normal Strain:\\
\vspace{1mm}
\tab $\ds \epsilon_{x} = -\frac{y}{p} = -\frac{y}{c}\epsilon_{\text{max}}$

Normal Stress:\\
\vspace{1mm}
\tab $\ds \sigma = -\frac{y}{c}\sigma_{\text{max}}$\\
\vspace{1mm}
\tab $\ds \sigma = \frac{My}{I}$\\
\vspace{1mm}
\tab $\ds \sigma_{\text{max}} = \frac{Mc}{I}$

Second Moment of Inertia:\\
\tab $I = \int y^2\,dA$\\
\tab Circle: $\ds I = \frac{\pi}{4}r^4$\\
\tab Rectangle: $\ds I = \frac{1}{12}bh^3$

Neutral Axis:\\
\tab $\int y\,dA = 0$

Section Modulus:\\
\tab $S = I/c$

Parallel Axis Theorem:\\
\tab $I_{\parallel} = I_G + Md^2$

Composite Beams:\\
\tab $n = \dfrac{E_2}{E_1}$ for $E_2 > E_1$\\
\vspace{1mm}
\tab $\sigma_2 = n \sigma_1$

\section{Stress Transformations}
\tab $\ds \sigma_x' = \frac{\sigma_x + \sigma_y}{2}	+ \frac{\sigma_x - \sigma_y}{2}\cos2\theta + \tau_{xy}\sin2\theta$
\\[2mm]
\tab $\ds \sigma_y' = \frac{\sigma_x + \sigma_y}{2}	- \frac{\sigma_x - \sigma_y}{2}\cos2\theta - \tau_{xy}\sin2\theta$
\\[2mm]
\tab $\ds \tau_{x'y'} = -\frac{\sigma_x - \sigma_y}{2}\sin2\theta + \tau_{xy}\cos2\theta$

Principal Stresses:\\
\tab $\ds \sigma_{1,2} = \frac{\sigma_x + \sigma _y}{2} \pm \sqrt{\left(\frac{\sigma_x - \sigma_y}{2}\right)^2 + \tau_{xy}^2}$

Maximum In-Plane Shear Stress:\\
\vspace{1mm}
\tab $\ds \tau_{\text{max}} = R = \sqrt{\left(\frac{\sigma_x - \sigma_y}{2}\right)^2 + \tau_{xy}^2}$

Angle of Principal In-Plane Stresses:\\
\tab $\ds \tan 2\theta_p = \frac{2\tau_{xy}}{(\sigma_x - \sigma_y)}$

Angle of Maximum In-Plane Stresses:\\
\tab $\ds \tan 2\theta_s = -\frac{(\sigma_x - \sigma_y)}{2\tau_{xy}}$

3D Analysis:\\
\tab $\tau_{\text{max}} = \frac{1}{2}|\sigma_{\text{max}} - \sigma_{\text{min}}|$

\section{Theories of Failure}
Maximum-Shearing-Stress Criterion for a yield strength $\sigma_Y$ and principal stresses $\sigma_a, \sigma_b$:\\
\tab Same sign: $|\sigma_a| < \sigma_Y$ \,and\, $|\sigma_b| < \sigma_Y$\\
\tab Opposite sign: $|\sigma_a - \sigma_b| < \sigma_Y$

Distortion Energy per unit volume:\\
\tab $\ds u_d = \frac{1}{6G}(\sigma^2_a - \sigma_a \sigma_b + \sigma_b^2)$

Maximum Distortion Energy Criterion:\\
\tab $\sigma^2_a - \sigma_a \sigma_b + \sigma_b^2 < \sigma_Y^2$


\section{Thin-Walled Pressure Vessel Stress}
\subsubsection{Cylindrical}
Hoop Stress:\\
\tab $\ds \sigma_1 = \frac{pr}{t}$

Longitudinal Stress:\\
\tab $\ds \sigma_2 = \frac{pr}{2t}$

Maximum In-Plane Shearing Stress:\\
\tab $\ds \tau_{\text{max}} = \frac{1}{2}\sigma_2 = \frac{pr}{4t}$

Maximum Out-of-Plane Shearing Stress (45\degree rotation around a longitudinal axis):\\
\tab $\ds \tau_{\text{max}} = \frac{pr}{2t}$

\subsubsection{Circle}
\tab $\ds \sigma_1 = \sigma_2 \frac{pr}{2t}$

Maximum In-Plane Shearing Stress:\\
\tab $\ds \tau_{\text{max}} = 0$ (reduces to a point)

Maximum Out-of-Plane Shearing Stress (45\degree rotation around a longitudinal axis):\\
\tab $\ds \tau_{\text{max}} = \frac{1}{2}\sigma_1 = \frac{pr}{4t}$

% Footer content
\rule{0.3\linewidth}{0.25pt}
\scriptsize\\
Updated \today\\
\href{https://github.com/DonneyF/formula-sheets}{https://github.com/DonneyF/formula-sheets}

\end{multicols}
\end{document}
