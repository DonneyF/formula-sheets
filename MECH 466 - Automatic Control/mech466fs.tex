% !Tex program = pdflatex

\documentclass[12pt,landscape]{article}
\usepackage{multicol}
\usepackage{calc}
\usepackage{ifthen}
\usepackage[landscape]{geometry}
\usepackage{amsmath,amsthm,amsfonts,amssymb}
\usepackage{color,graphicx,overpic}
\usepackage{hyperref}
\usepackage{enumitem}
\usepackage{upgreek}
\usepackage[italicdiff]{physics}
\usepackage{newtxtext,newtxmath}
\usepackage{booktabs}
\usepackage{mdframed}

% This sets page margins to .5 inch if using letter paper, and to 1cm
% if using A4 paper. (This probably isn't strictly necessary.)
% If using another size paper, use default 1cm margins.
\ifthenelse{\lengthtest { \paperwidth = 11in}}
	{ \geometry{top=.5in,left=.5in,right=.5in,bottom=.5in} }
	{\ifthenelse{ \lengthtest{ \paperwidth = 297mm}}
		{\geometry{top=1cm,left=1cm,right=1cm,bottom=1cm} }
		{\geometry{top=1cm,left=1cm,right=1cm,bottom=1cm} }
	}

% Turn off header and footer
\pagestyle{empty}
 

% Redefine section commands to use less space
\makeatletter
\renewcommand{\section}{\@startsection{section}{1}{0mm}%
                                {-1ex plus -.5ex minus -.2ex}%
                                {0.5ex plus .2ex}%x
                                {\normalfont\normalsize\bfseries}}
\renewcommand{\subsection}{\@startsection{subsection}{2}{0mm}%
                                {-1explus -.5ex minus -.2ex}%
                                {0.5ex plus .2ex}%
                                {\normalfont\small\bfseries}}
\renewcommand{\subsubsection}{\@startsection{subsubsection}{3}{0mm}%
                                {-1ex plus -.5ex minus -.2ex}%
                                {1ex plus .2ex}%
                                {\normalfont\footnotessize\bfseries}}
\makeatother

% Define BibTeX command
\def\BibTeX{{\rm B\kern-.05em{\sc i\kern-.025em b}\kern-.08em
    T\kern-.1667em\lower.7ex\hbox{E}\kern-.125emX}}

% Don't print section numbers
\setcounter{secnumdepth}{0}


\setlength{\parindent}{0pt}
\setlength{\parskip}{1pt plus 0.5ex}

\newcommand{\tab}{\hspace{.02\textwidth}}
\newcommand{\ds}{\displaystyle}

% Redefine some commands for newtxmath boldness
\renewcommand{\grad}{\nabla}
\renewcommand{\curl}[1]{\nabla\times#1}
\renewcommand{\div}[1]{\nabla\cdot#1}
\renewcommand{\cross}{\times}

\newcommand{\Var}[1]{\mathrm{Var}(#1)}
\newcommand{\Cov}[1]{\mathrm{Cov}(#1)}

\mdfsetup{skipabove=2pt,skipbelow=2pt, innertopmargin=-6pt, innerbottommargin=2pt, innerleftmargin=2pt, innerrightmargin=2pt}
\theoremstyle{definition}
\newmdtheoremenv{theorem}{Theorem}

% -----------------------------------------------------------------------

\begin{document}

\raggedright
\footnotesize
\begin{multicols}{3}


% multicol parameters
% These lengths are set only within the two main columns
%\setlength{\columnseprule}{0.25pt}
\setlength{\premulticols}{1pt}
\setlength{\postmulticols}{1pt}
\setlength{\multicolsep}{1pt}
\setlength{\columnsep}{2pt}

\raggedcolumns

\begin{center}
	\Large{\underline{MECH 466 Formula Sheet}}
\end{center}

\section{Continuous Time Signals}
Linearity:\\
\tab $\mathcal{S}[\alpha x(t) + \beta y(t)] = \alpha\mathcal{S}[x(t)] + \beta\mathcal{S}[y(t)]$

Time Invariance:\\
\tab If $\mathcal{S}[x(t)] = y(t)$ then $\mathcal{S}[x(t \pm \tau)] = y(t \pm \tau)$.

\section{Laplace Transform}
Initial Value Theorem:\\
\tab $\ds f(0+) \Leftrightarrow \lim_{s\rightarrow\infty} sF(s)$

Final Value Theorem:\\
\tab $\ds \lim_{t\rightarrow\infty} f(t) \Leftrightarrow \lim_{s\rightarrow 0} sF(s)$

\section{Stability}
\begin{itemize}
	\itemsep 0em
	\item BIBO stability: any bounded input provides a bounded output
	\item Asymptotic stability: Initial conditions generates $y(t)$ converges to zero.
\end{itemize}

Characteristic Equation:\\
\tab For $G(s) = N(s)/D(s)$, the characteristic equation is
$D(s) = 0$.

Stability Condition in $s$-Domain:\\
\tab All poles in open left hand plane $\iff$ System is BIBO and asymptotically stable\\

Marginal Stability:\\
\begin{itemize}
	\itemsep 0em
	\item $G(s)$ has no pole in the open RHP
	\item $G(s)$ has at least one simple pole on the imaginary axis
	\item $G(s)$ has no repeated pole on the imaginary axis
	\item BIBO stable except for sinusoidal inputs.
	\item For any non-zero initial condition, the output neither converges to zero nor diverge.
\end{itemize}

Polynomials:\\
\begin{itemize}
	\itemsep 0em
	\item For 1st and 2nd order polynomials, all roots are in LHP $\iff$ coefficients have the same sign
	\item For 3rd and higher orders, all roots are in LHP $\implies$ coefficients have the same sign
\end{itemize}

\subsection{Routh-Hurwitz Criterion}
\begin{itemize}
	\item The number of roots in the open right half-plane is equal to the number of sign changes in the first column of Routh array.
	\item If zero row appears in the Routh array, roots are in imaginary axis or RHP.
	\item If zero row appears, replace the zero with the coefficients of derivative of auxiliary polynomial.
	\item Auxiliary polynomial: The polynomial above the zero row. 
\end{itemize}

\section{Steady State Error}
Step Function:\\
\tab $\ds r(t) = Ru(t) \implies e_\text{ss} = \frac{R}{1+K_p} = \frac{R}{1+L(0)}$

Ramp Function:\\
\tab $\ds r(t) = Rtu(t) \implies e_\text{ss} = \frac{R}{K_v} = \frac{R}{\lim\limits_{s\to 0}sL(s)}$

Parabolic Function:\\
\tab $\ds r(t) = \frac{Rt}{2}u(t) \implies e_\text{ss} = \frac{R}{K_a} = \frac{R}{\lim\limits_{s\to 0}s^2L(s)}$

\section{First Order System}
\tab $\ds G(s)=\frac{K}{Ts+1}$

Step Response:\\
\tab $y(t) = K(1-e^{t/T})u(t)$

Time Constant:\\
\tab $\ds T = \frac{1}{|\text{Real part of pole}|}$\\
\tab Response is slower the closer the pole is to the imaginary axis

Settling Time:\\
\tab 5\%: $\approx 3T$
\tab 2\%: $\approx 4T$

\section{Second Order System}
\tab $\ds G(s) = \frac{\omega_n^2}{s^2 + 2\zeta\omega_n s + \omega_n^2}$

Underdamped Step Response:\\
\tab $\ds y(t) = 1-\frac{e^{-\zeta\omega_n t}}{\sqrt{1-\zeta^2}}\sin\left(\omega_d t + \arccos\zeta\right)$\\
\tab $\omega_d = \omega_n\sqrt{1-\zeta^2}$

Poles:\\
\tab $s = -\zeta\omega_n \pm j\omega_n\sqrt{1-\zeta^2}$

Settling Time:\\
\tab 5\%: $\approx 3/\zeta\omega_n$
\tab 2\%: $\approx 4/\zeta\omega_n$

Peak Time:\\
\tab $T_p = \pi/\omega_d$

Overshoot:\\
\tab 16\%: $\zeta = 0.5$
\tab 5\%: $\zeta = \sqrt{2}/2$\\
\tab $\ds y_\text{max} = 1+e^{-\zeta\pi/\sqrt{1-\zeta^2}}$\\
\tab Percent overshoot = $100 e^{-\zeta\pi/\sqrt{1-\zeta^2}}$

\end{multicols}

\newpage

\begin{minipage}[t]{0.27\textwidth}
\section{Reserved}

% Footer content
\rule{0.3\linewidth}{0.25pt}
\scriptsize\\
Updated \today\\
\href{https://github.com/DonneyF/formula-sheets}{https://github.com/DonneyF/formula-sheets}
\end{minipage}%
\hspace{0.5em}
\begin{minipage}[t]{0.35\textwidth}
\subsection{Reserved}
\end{minipage}
\end{document}
