% !Tex program = pdflatex

\documentclass[12pt,landscape]{article}
\usepackage{multicol}
\usepackage{calc}
\usepackage{ifthen}
\usepackage{mathtools}
\usepackage[landscape]{geometry}
\usepackage{amsmath,amsthm,amsfonts,amssymb}
\usepackage{color,graphicx,overpic}
\usepackage{hyperref}
\usepackage{enumitem}
\usepackage{upgreek}
\usepackage[italicdiff]{physics}
\usepackage{newtxtext,newtxmath}
\usepackage{mdframed}
\usepackage{amsbsy}

% This sets page margins to .5 inch if using letter paper, and to 1cm
% if using A4 paper. (This probably isn't strictly necessary.)
% If using another size paper, use default 1cm margins.
\ifthenelse{\lengthtest { \paperwidth = 11in}}
	{ \geometry{top=.5in,left=.5in,right=.5in,bottom=.5in} }
	{\ifthenelse{ \lengthtest{ \paperwidth = 297mm}}
		{\geometry{top=1cm,left=1cm,right=1cm,bottom=1cm} }
		{\geometry{top=1cm,left=1cm,right=1cm,bottom=1cm} }
	}

% Turn off header and footer
\pagestyle{empty}
 

% Redefine section commands to use less space
\makeatletter
\renewcommand{\section}{\@startsection{section}{1}{0mm}%
                                {-1ex plus -.5ex minus -.2ex}%
                                {0.5ex plus .2ex}%x
                                {\normalfont\normalsize\bfseries}}
\renewcommand{\subsection}{\@startsection{subsection}{2}{0mm}%
                                {-1explus -.5ex minus -.2ex}%
                                {0.5ex plus .2ex}%
                                {\normalfont\small\bfseries}}
\renewcommand{\subsubsection}{\@startsection{subsubsection}{3}{0mm}%
                                {-1ex plus -.5ex minus -.2ex}%
                                {1ex plus .2ex}%
                                {\normalfont\footnotessize\bfseries}}
\makeatother

% Define BibTeX command
\def\BibTeX{{\rm B\kern-.05em{\sc i\kern-.025em b}\kern-.08em
    T\kern-.1667em\lower.7ex\hbox{E}\kern-.125emX}}

% Don't print section numbers
\setcounter{secnumdepth}{0}


\setlength{\parindent}{0pt}
\setlength{\parskip}{1pt plus 0.5ex}

\newcommand{\tab}{\hspace*{1em}}
\newcommand{\ds}{\displaystyle}

% Redefine some commands for newtxmath boldness
\renewcommand{\grad}{\nabla}
\renewcommand{\curl}[1]{\nabla\times#1}
\renewcommand{\div}[1]{\nabla\cdot#1}
\renewcommand{\cross}{\times}
\newcommand{\defn}[1]{\textbf{Def} (\emph{#1})}
\newcommand{\thm}[1]{\textbf{Thm} (\emph{#1})}

\newcommand{\Var}[1]{\mathrm{Var}(#1)}
\newcommand{\Cov}[1]{\mathrm{Cov}(#1)}

\mdfsetup{skipabove=2pt,skipbelow=2pt, innertopmargin=-6pt, innerbottommargin=2pt, innerleftmargin=2pt, innerrightmargin=2pt}
\theoremstyle{definition}
\newmdtheoremenv{theorem}{Theorem}

% -----------------------------------------------------------------------

\begin{document}

\raggedright
\footnotesize
\begin{multicols*}{3}

\raggedcolumns

% multicol parameters
% These lengths are set only within the two main columns
%\setlength{\columnseprule}{0.25pt}
\setlength{\premulticols}{1pt}
\setlength{\postmulticols}{1pt}
\setlength{\multicolsep}{1pt}
\setlength{\columnsep}{2pt}

\begin{center}
	\Large{\underline{MATH 400 Formula Sheet}}
\end{center}

\section{Linear First Order Equations}
First Order Linear PDE:\\
\tab $a(x,y)u_x + b(x,y)u_y + c(x,y)u = f(x,y)$

\subsection{Simple Transport Equation}
$$u_t + cu_x = 0 \,, -\infty < x < \infty, t > 0$$
$$u(x,0) = \phi(x) \,, -\infty < x < \infty$$
Solution:\\
\tab $u(x,t) = \phi(x - ct)$

\section{Linear Second Order Equations}
Quasilinear PDE:\\
\tab $a_{11}(x, y, u, u_x, u_y)u_{xx} + 2a_{12}(x, y, u, u_x, u_y)u_{xy} + a_{22}(x, y, u, u_x, u_y)u_{yy} + a_{00}(x, y, u, u_x, u_y) = 0$

Semilinear PDE:\\
\tab $a_{11}(x, y)u_{xx} + 2a_{12}(x, y)u_{xy} + a_{22}(x, y)u_{yy} + a_{00}(x, y, u, u_x, u_y) = 0$

Linear PDE:\\
\tab $a_{11}(x, y)u_{xx} + 2a_{12}(x, y)u_{xy} + a_{22}(x, y)u_{yy} + a_{1}(x,y)u_x +  a_{2}(x,y)u_y + a_{0}(x,y)u = f(x,y)$

Discriminant for semilinear PDEs:\\
\tab $\mathcal{D}(x,y) = [a_{12}(x,y)]^2 - a_{11}(x,y)a_{22}(x,y)$

Classification of semilinear PDEs:
\begin{equation*}
\begin{cases}
\mathcal{D}(x,y) > 0 & \text{Hyperbolic}\\
\mathcal{D}(x,y) < 0 & \text{Elliptic}\\
\mathcal{D}(x,y) = 0 & \text{Parabolic}
\end{cases}
\end{equation*}

Change of variables:
\begin{equation*}
\begin{cases}
U_{\xi\eta} + b_{00}(\xi, \eta, U, U_\xi, U_\eta)  = 0 & \text{Hyperbolic}\\
U_{\xi\xi} + U_{\eta\eta} + b_{00}(\xi, \eta, U, U_\xi, U_\eta)  = 0 & \text{Elliptic}\\
U_{\xi\xi} + b_{00}(\xi, \eta, U, U_\xi, U_\eta)= 0 & \text{Parabolic}
\end{cases}
\end{equation*}

\subsection{Wave Equation}
General Solution:\\
\tab $u(x,t) = f(x + ct) + g(x-ct)$

Initial Value Problem on the real line:\\
\begin{equation*}
\begin{cases}
u_{tt} - c^2u_{xx} = f(x,t) \,, -\infty < x < \infty, t > 0 &\\
u(x,0) = \phi(x) \,, u_t(x,0) = \psi(x) -\infty < x < \infty&\\
\end{cases}
\end{equation*}

d'Alembert's Formula:\\
\tab $\ds u(x,t) = \frac{1}{2}[\phi(x-ct) + \phi(x+ct)] + \frac{1}{2c}\int_{x-ct}^{x+ct}\psi(s)\mathop{ds} + \frac{1}{2c} \int_{0}^{t}\int_{x-c(t-s)}^{x+c(t-s)}f(y,s)\mathop{dy}\mathop{ds}$

\subsection{Non-Homogenous Boundary Conditions}
\tab Shift the data ($u = v + w$):

Dirichlet, $u(0,t) = a(t)$, $u(L,t) = b(t)$:\\
\tab $\ds w(x,t) = a(t) + \frac{x}{L}[b(t) - a(t)]$

Neumann, $u_x(0,t) = a(t)$, $u_x(L,t) = b(t)$\\
\tab $\ds w(x,t) = xa(t) + \frac{x^2}{2L}[b(t) - a(t)]$

Mixed 1, $u(0,t) = a(t)$, $u_x(L,t) = b(t)$\\
\tab $w(x,t) = a(t) + xb(t)$

Mixed 2, $u_x(0,t) = a(t)$, $u(L,t) = b(t)$\\
\tab $w(x,t) = (x-L)a(t) + b(t)$

\subsection{Sturm-Liouville Theory}
Consider the homogeneous linear second-order PDE:\\
\tab $r(x)u_t - (p(x)u_x)_x + q(x)u = 0$

Separation of Variables:\\
\tab $\ds -\frac{T'(t)}{T(t)} = -\frac{(p(x)X'(x))'}{r(x)X(x)} + \frac{q(x)}{r(x)} = \lambda$

ODEs:
\begin{equation*}
\begin{cases}
T' + \lambda T = 0 &\\
-(p(x)X')' + q(x)X = \lambda r(x)X &\\
\end{cases}
\end{equation*}

Associated Eigenvalue Problem:\\
\tab $-(p(x)X')' + q(x)X = \lambda r(x) X$

General Boundary Conditions:\\
\tab $\mathcal{B}_1X = \alpha_1 X(a) + \beta_1 X(b) + \gamma_1 X'(a) + \delta_1 X'(b) = 0$\\
\tab $\mathcal{B}_2X = \alpha_2 X(a) + \beta_2 X(b) + \gamma_2 X'(a) + \delta_2 X'(b) = 0$

Boundary conditions are separated when:\\
\tab $\beta_1 = \delta_1 = 0 \hspace{1cm} \alpha_2 = \gamma_2 = 0$

Lagrange's Identity. BCs are symmetric if for all functions $f$, $g$ that satisfy the BCs:\\
\tab $\tab \left[-p\left(f'g - fg'\right)\right]_a^b = 0$

Sturm-Liouville Problem ($p(x) > 0, r(x) > 0$):
\begin{equation*}
\begin{cases}
\mathcal{L}X(x) = \lambda r(x)X, \, a < x < b &\\
\mathcal{B}_1 X = 0, \,\mathcal{B}_2 X = 0&\\
\end{cases}
\end{equation*}

Theorem 5.3.1:\\
\tab If you have symmetric BCs, then any to eigenfunctions of a SL problem that correspond to distinct eigenvalues are orthogonal. If any function is expanded in a series of these eigenfunctions, the coefficients are determined.

Theorem 5.3.2:\\
\tab Under Theorem 5.3.1, all the eigenvalues are real numbers and the eigenfunctions can be chosen to be real-valued.

Theorem 5.3.3:\\
\tab Under 5.3.1, if $q(x) \geq 0$ for all $a \leq x \leq b$ and if for all real-valued functions $f$ satisfying the BCs we satisfy
$$[p(x)f(x)f'(x)]_a^b \leq 0$$
then there can be no negative eigenvalues.

Theorem 5.4.1:\\
\tab For any regular of periodic SL problem, there are an infinite amount of eigenvalues.

\section{Laplace's Equation}

Polar Coordinates:\\
\tab $\ds u_{rr} + \frac{1}{r}u_r + \frac{1}{r^2} u_{\theta \theta} = 0$

Spherical Coordinates:\\
\tab $\ds u_{rr} + \frac{2}{r}u_r + \frac{1}{r^2 \sin\theta}\left[\sin\theta \,u_\theta\right]_\theta + \frac{1}{r^2\sin^2\theta}u_{\phi\phi} = 0$

Maximum Principle:\\
\tab Let $D$ be a connected bounded open set of $\mathbb{R}^n$, $n = 2, 3$. If $u(\vb{x})$ is harmonic in $D$ and continuous on $D \cup \partial D$ then $u(\vb{x})$ attains its maximum and minimum values on $\partial D$.

\section{PDEs in 3 Dimensions}
Elliptic PDE:\\
\tab $-\div[p(\vb{x})\grad{u}] + q(\vb{x}) u = f(\vb{x})$

Parabolic PDE:\\
\tab $r(\vb{x})u_t -\div[p(\vb{x})\grad{u}] + q(\vb{x}) u = f(\vb{x}, t)$

Hyperbolic PDE:\\
\tab $r(\vb{x})u_{tt} -\div[p(\vb{x})\grad{u}] + q(\vb{x}) u = f(\vb{x}, t)$

Eigenvalue Problem:\\
\tab $-\div[p(\vb{x})\grad{X}] + q(\vb{x}) X = \lambda r(\vb{x})X$

Green's First Identity:\\
\tab Let $D$ be an open connected domain in $\mathbb{R}^3$ and let $\partial D$ be its piecewise smooth boundary Let $f$ and $g$ be smooth functions on $D \cup \partial D$. Then\\
\tab $\ds \oint_{\partial D} f \pdv{g}{n}\mathop{ds} = \iint_{D}\grad{F}\cdot\grad{g}\mathop{d\vb{x}} + \iint_{D} f \grad^2{g}\mathop{d\vb{x}}$

Bessel's Equation:\\
\tab $\rho^2 y'' + \rho y + (\rho^2 -n^2)y = 0$\\
\tab $y(\rho) = c_1 J_n(\rho) + c_2 Y_n(\rho)$

Associated Legendre Equation:\\
\tab $\ds \dv{x}\left[(1-x^2)\dv{x}P_l^m(x)\right] + \left[l(l+1) - \frac{m^2}{1-x^2}\right]P_l^m(x)=0$

Spherical Harmonics:\\
\tab $\ds \grad^2_{S^2} Y_l^m = \frac{1}{\sin\theta}[\sin\theta\,Y]_\theta + \frac{1}{\sin^2\theta}Y_{\phi\phi} = l(l+1)Y_l^m$
\tab $Y_l^m(\theta,\phi) = P_l^\abs{m}(\cos\theta)e^{im\phi}$
\tab $l = 0, 1, 2 \ldots \quad m = -l, -l+1,\ldots, l-1, l$

\section{Integral Transform Methods}
Fourier Transform:\\
\tab $\ds \hat{f}(k) = \int_{-\infty}^{\infty} f(x)e^{-ikx}\mathop{dx}$

Inverse Fourier Transform:\\
\tab $\ds \hat{f}(k) = \frac{1}{2\pi}\int_{-\infty}^{\infty} f(x)e^{ikx}\mathop{dx}$

Error Function:\\
\tab $\ds \erf(x) = \frac{2}{\sqrt{\pi}}\int_{0}^{x} e^{-p^2} \mathop{dp}$
\tab $\erf(0) = 0 \quad \lim_{x\rightarrow\infty}\erf(x) = 1$

Laplace Transform:\\
\tab Let $f(t)$ be a piecewise continuous function on $[0,\infty)$ and suppose $\exists K, \gamma$ such that $\abs{f(t)} \leq K e^{\gamma t} \, \forall t > 0$. Then
\tab $\ds F(s) = \int_{0}^{\infty}f(t)e^{-st}\mathop{dt}$

Inverse Laplace Transform:\\
\tab $\ds f(t) = \frac{1}{2\pi i}\int_{\alpha - i\infty}^{\alpha + i \infty}F(s)e^{st}\mathop{ds} \quad \alpha > \gamma$




\newpage

\section{Solved 1D Eigenvalue Problems}
\subsection{Dirichlet}
\begin{equation*}
	\begin{rcases}
		-X'' = \lambda X\\
		X(0) = 0, X(L) = 0
	\end{rcases} \implies \begin{cases}
		\lambda_n = \frac{n\pi}{L}, \, n = 1, 2, 3 \ldots\\
		X_n(x) = \sin(\frac{n\pi}{L}x) 
	\end{cases}
\end{equation*}

\subsection{Neumann}
\begin{equation*}
	\begin{rcases}
		-X'' = \lambda X\\
		X'(0) = 0, X'(L) = 0
	\end{rcases} \implies \begin{cases}
		\lambda_n = \frac{n\pi}{L}, \, n = 0, 1, 2 \ldots\\
		X_n(x) = \cos(\frac{n\pi}{L}x) 
	\end{cases}
\end{equation*}

\subsection{Periodic}
\begin{equation*}
	\begin{rcases}
		-X'' = \lambda X\\
		X(-\pi) = X(\pi)\\
		X'(-\pi) = X'(\pi)
	\end{rcases} \implies \begin{cases}
		\lambda_n = n^2, \, n = 0, 1, 2 \ldots\\
		X_n(x) \in \{1, \cos(n\theta), \sin(n\theta)\}
	\end{cases}
\end{equation*}

\subsection{Mixed 1}
\begin{equation*}
	\begin{rcases}
		-X'' = \lambda X\\
		X(0) = 0, X'(L) = 0
	\end{rcases} \implies \begin{cases}
		\lambda_n = \frac{(2n+1)\pi}{2L}, \, n = 0, 1, 2 \ldots\\
		X_n(x) = \sin(\frac{(2n+1)\pi}{2L}x)
	\end{cases}
\end{equation*}

\subsection{Mixed 2}
\begin{equation*}
	\begin{rcases}
		-X'' = \lambda X\\
		X'(0) = 0, X(L) = 0
	\end{rcases} \implies \begin{cases}
		\lambda_n = \frac{(2n+1)\pi}{2L}, \, n = 0, 1, 2 \ldots\\
		X_n(x) = \cos(\frac{(2n+1)\pi}{2L}x)
	\end{cases}
\end{equation*}


\section{Solved PDE Problems}
\subsection{Problem 1}
$$\Delta u = 0 \quad x^2 + y^2 < a^2$$
$$u = h(x,y) \quad \text{on} \quad x^2 + y^2 = a^2$$
Solution (Oct 21):\\
\tab $\ds u(r,\theta) = \frac{a^2 - r^2}{2\pi} \int_{-\pi}^{\pi} \frac{h(\phi)}{a^2 - 2ar\cos(\phi - \theta) + r^2}\mathop{d\phi}$

\subsection{Problem 2}
$$\Delta u = 0 \quad 0 < r < a, \quad 0 < \theta < \beta$$
$$u(r, 0) = 0, \quad u(r, \beta) = 0, \quad 0 < r < a$$
$$u_r(a,\theta) = h(\theta) \quad 0 < \theta < \beta$$
Solution (Oct 23):\\
\tab $\ds u(r,\theta) = \sum_{n=1}^{\infty} C_n r^{n\pi/\beta}\sin\left(\frac{n\pi}{\beta}\theta\right)$\\
\tab $\ds C_n = \frac{2}{n\pi}a^{-n\pi/\beta + 1}\int_{0}^{\beta}h(\phi)\sin\left(\frac{n\pi}{\beta}\theta\right)$

\subsection{Problem 3}
$$\Delta u = 0 \quad a < r < b, \quad 0 < \theta < 2\pi$$
General Solution (Oct 23):\\
\tab $\ds u(r, \theta) = \frac{1}{2} \left(C_0 + D_0\ln(r)\right) + \sum_{n=1}^{\infty}\left(C_n^{(1)}r^n + D_n^{(1)} r^{-n}\right)\cos(n\theta) + \left(C_n^{(2)}r^n + D_n^{(2)}r^{-n}\right)\sin(n\theta)$

\subsection{Problem 4}
$$S_t - DS_{xx}=0 \quad -\infty < x < \infty, \quad t > 0$$
$$S(x, 0) = \delta(x - y), \quad y \geq 0$$
Solution (Nov 16):\\
\tab $\ds S(x,t) = \frac{1}{\sqrt{4\pi Dt}}e^{-(x - y)^2/(4Dt)}$

\subsection{Problem 5}
$$u_t - Du_{xx}=0 \quad -\infty < x < \infty, \quad t > 0$$
$$u(x, 0) = \varphi(x), \quad -\infty < x < \infty$$
Solution (Nov 18):\\
\tab $\ds u(x,t) =\int_{-\infty}^{\infty} \varphi(y)S(x - y, t)\mathop{dy}$

\subsection{Problem 6}
$$u_t - Du_{xx}=0 \quad 0 < x < \infty, \quad t > 0$$
$$u(0, t) = 0, \quad t > 0$$
$$u(x, 0) = \varphi(x), \quad -\infty < x < \infty$$
Solution (Nov 18):\\
\tab $\ds u(x,t) =\int_{0}^{\infty} \phi(y)[S(x-y, t) - S(x + y, t)]\mathop{dy}$

\subsection{Problem 7}
$$\Delta u =0 \quad -\infty < x < \infty, \quad y > 0$$
$$u(x,0) = h(x), \quad -\infty < x < \infty$$
Solution (Nov 20):\\
\tab $\ds u(x,y) = \int_{-\infty}^{\infty}h(z) \frac{y}{\pi [(x - z)^2 - y^2]}\mathop{dz}$

\subsection{Problem 8}
$$u_t - Du_{xx}=0 \quad 0 < x < \infty, \quad t > 0$$
$$u(0, t) = h(t), \quad t > 0$$
$$u(x, 0) = 0, \quad -\infty < x < \infty$$
Solution (Nov 23):\\
\tab $\ds u(x,t) = \begin{cases}
	0 & t < b\\
	1 - \erf\left(\frac{x}{\sqrt{4D(t-b)}}\right) & t > b
\end{cases}$

% Footer content
\rule{0.3\linewidth}{0.25pt}
\scriptsize\\
Updated \today\\
\href{https://github.com/DonneyF/formula-sheets}{https://github.com/DonneyF/formula-sheets}
\end{multicols*}%

\end{document}
