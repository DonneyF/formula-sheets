\documentclass[12pt,landscape]{article}
\usepackage{multicol}
\usepackage{calc}
\usepackage{ifthen}
\usepackage[landscape]{geometry}
\usepackage{amsmath,amsthm,amsfonts,amssymb}
\usepackage{color,graphicx,overpic}
\usepackage{hyperref}
\usepackage{enumitem}
\usepackage{physics}
\usepackage{newtxtext,newtxmath}

% This sets page margins to .5 inch if using letter paper, and to 1cm
% if using A4 paper. (This probably isn't strictly necessary.)
% If using another size paper, use default 1cm margins.
\ifthenelse{\lengthtest { \paperwidth = 11in}}
{ \geometry{top=.5in,left=.5in,right=.5in,bottom=.5in} }
{\ifthenelse{ \lengthtest{ \paperwidth = 297mm}}
	{\geometry{top=1cm,left=1cm,right=1cm,bottom=1cm} }
	{\geometry{top=1cm,left=1cm,right=1cm,bottom=1cm} }
}

% Turn off header and footer
\pagestyle{empty}

% Redefine section commands to use less space
\makeatletter
\renewcommand{\section}{\@startsection{section}{1}{0mm}%
	{-1ex plus -.5ex minus -.2ex}%
	{0.5ex plus .2ex}%x
	{\normalfont\large\bfseries}}
\renewcommand{\subsubsection}{\@startsection{subsection}{2}{0mm}%
	{-1explus -.5ex minus -.2ex}%
	{0.5ex plus .2ex}%
	{\normalfont\normalsize\bfseries}}
\renewcommand{\subsubsection}{\@startsection{subsubsection}{3}{0mm}%
	{-1ex plus -.5ex minus -.2ex}%
	{1ex plus .2ex}%
	{\normalfont\small\bfseries}}
\makeatother

% Define BibTeX command
\def\BibTeX{{\rm B\kern-.05em{\sc i\kern-.025em b}\kern-.08em
		T\kern-.1667em\lower.7ex\hbox{E}\kern-.125emX}}

% Don't print section numbers
\setcounter{secnumdepth}{0}

\setlength{\parindent}{0pt}
\setlength{\parskip}{1pt plus 0.5ex}

% Custom Commands
\newcommand{\ds}{\displaystyle}
\newcommand{\tab}{\hspace{.02\textwidth}}
% -----------------------------------------------------------------------

\begin{document}
\raggedright
\footnotesize
\begin{multicols}{3}
	
	
% multicol parameters
% These lengths are set only within the two main columns
%\setlength{\columnseprule}{0.25pt}
\setlength{\premulticols}{1pt}
\setlength{\postmulticols}{1pt}
\setlength{\multicolsep}{1pt}
\setlength{\columnsep}{2pt}

\begin{center}
	\Large{\underline{MATH 217 Formula Sheet}}
\end{center}

\section{Vectors \& Geometry of Space}
Distance between two points:\\
\tab $d = \sqrt{(x_1-x_0)^2 + (y_1-y_0)^2 + (z_1 - z_0)^2}$

Equation of a sphere:\\
\tab $R^2 = (x_1-x_0)^2 + (y_1-y_0)^2 + (z_1 - z_0)^2$

Unit vector:\\
\tab $\vb{u} = \frac{\vb{a}}{|\vb{a}|}$

Dot Product:\\
\tab $\vb{a}\cdot\vb{b}=\|\vb{a}\|\ \|\vb{b}\|\cos(\theta)$

Scalar projection of $\vb{a}$ on to $\vb{b}$:\\
\tab $\text{comp}_{\vb{b}}\vb{a}=|\vb{a} |\cos \theta =|\vb{a} |{\frac {\vb{a} \cdot \vb{b} }{|\vb{a} |\,|\vb{b} |}}={\frac {\vb{a} \cdot \vb{b} }{|\vb{b} |}}$

Vector projection of $\vb{a}$ on to $\vb{b}$:\\
\tab $\text{proj}_{\vb{b}}\vb{a} ={\frac {\vb{a} \cdot \vb{b} }{|\vb{b} |}}{\frac {\vb{b} }{|\vb{b} |}}$

Orthogonal projection of $\vb{a}$ on to $\vb{b}$:\\
\tab $\vb{v} = \vb{a} - \text{proj}_{\vb{b}}\vb{a}$

Cross product:\\
\tab $\vb{a} \times \vb{b} =\left\|\vb{a} \right\|\left\|\vb{b} \right\|\sin(\theta )$

\subsubsection{Lines \& Planes}

Vector equation of a line:\\
\tab $\vb{r}(t) = \vb{r_0} + \vb{v}(t)$

Vector equation of a plane:\\
\tab $\vb{r} - \vb{r}_0 \cdot \vb{n} = 0$

Scalar equation of a plane, where $a,b,c$ are components of the normal vector:\\
\tab $a(x-x_0) + b(y-y_0) + c(z-z_0) = 0$

Distance from the point $P=(x_1,y_1,z_1)$ to the plane $Ax+By+Cz+D=0$:\\
\tab $d = \frac{|Ax_1+By_1+Cz_1 +D|}{\sqrt{A^2+B^2+C^2}}$

\subsubsection{Quadric Surfaces}

Equation of an ellipsoid:\\
\tab ${x^2 \over a^2} + {y^2 \over b^2} + {z^2 \over c^2} = 1$

Equation of an elliptic paraboloid:\\
\tab ${x^2 \over a^2} + {y^2 \over b^2} =z$

Equation of a hyperbolic paraboloid:\\
\tab ${x^2 \over a^2} - {y^2 \over b^2} = z$

Equation of a elliptic hyperboloid of one sheet:\\
\tab ${x^2 \over a^2} + {y^2 \over b^2} - {z^2 \over c^2} = 1$

Equation of an elliptic hyperboloid of two sheets:\\
\tab ${x^2 \over a^2} + {y^2 \over b^2} - {z^2 \over c^2} = - 1$

\section{Vector Functions}
Arc length of a vector function:\\
\tab $\int_a^b|\vb{r}'(t)|\,dt$

Length of the curve $y=f(x)$:\\
\tab $\int_a^b \sqrt{1+(f'(x))^2}\,dx$

If $\vb{r}'(t)$ is differentiable at $t=a$ and $ \vb{r}'(a) \neq \vb{r}(0)$, the tangent line to the curve given by $\vb{r}'(t)$ is the line through $\vb{r}'(a)$ in the direction of $\vb{r}'(a)$

\section{Partial Derivatives}
$f(x,y)$ is continuous at $(a,b)$ if\\
\tab $\lim_{(x,y)\to (a,b)}f(x,y)=f(a,b)$\\

Suppose that $z=f(x,y)$, $f$ is differentiable, $x=g(t)$, and $y=h(t)$. Assuming that the relevant derivatives exist,\\
\tab $\pdv{z}{t} = \pdv{z}{x}\pdv{x}{t}+\pdv{z}{y}\pdv{y}{t}$

If $f(x,y)$ is defined on a domain $D$ that contains the point $(a,b)$. If $\pdv{z}{y}{x}$ and $\pdv{z}{x}{y}$ are continuous on $D$, then\\
\tab $\pdv{z}{y}{x} = \pdv{z}{x}{y}$
	
Equation of a tangent plane:\\
\tab $z=f_x(x_0,y_0)(x-x_0)+f_y(x_0,y_0)(y-y_0)+f(x_0,y_0)$

The differential of $f(x,y)$ is:\\
\tab $df = \pdv{f}{x}dx + \pdv{f}{y}dy$

Suppose that $f(x,y)$ is a function and $x=g(s,t)$ and $y=h(s,t)$:\\
\tab $\pdv{f}{s}=f_xg_s+f_yh_s\qquad \pdv{f}{t}=f_xg_t+f_yh_t.$

The slope of a surface given by $z=f(x,y)$ in the direction of a vector $\vb{u}$ is called the directional derivative of $f$, written $D_uf$\\
\tab $D_{\vb{u}}f=\nabla f\cdot \vb{u}=|\nabla f||\vb{u}|\cos\theta=
	|\nabla f|\cos\theta$

The maximum value of the directional derivative $D_{\vb{u}} f(\vb{v})$ is $|\nabla f(\vb{v})|$and it occurs when $\vb{u}$ has the same direction as the gradient vector $\nabla f(\vb{v})$.\\
\vspace{0.2cm}
The gradient vector $\nabla f(a,b,c)$ is orthogonal to the level surface $S$ through $(a,b,c)$\\
\vspace{0.2cm}
Discriminant of $f(x,y)$:\\
\tab $D(x,y) = \det\mqty[f_{xx}& f_{xy}\\f_{yx} & f_{yy} ]$\\
\vspace{0.2cm}
If $f_{xx} > 0$ or $f_{yy} > 0$ and $D(a,b) > 0$, then $f(a,b)$ is a local minimum.\\
If $f_{xx} < 0 $ or $f_{yy} < 0$ and $D(a,b) > 0$, then $f(a,b)$ is a local maximum.\\
If $D(a,b) < 0$, then $f(a,b)$ is a saddle point.\\
If $D(a,b) = 0$, no information is given.\\

\subsubsection{Optimization}
The extreme values of $f(x,y)$ can only occur at:\\
\begin{itemize}
	\vspace{-0.5em}
	\setlength\itemsep{-0.3em}
	\item Interior critical points, where both partials exist.
	\item Boundary points of the domain of the function.
\end{itemize}
\vspace{-0.25cm}
To maximize or minimize $f(x,y)$ subject to the constraint $g(x,y) = C$, we solve:
\begin{itemize}
	\vspace{-0.5em}
	\setlength\itemsep{-0.3em}
	\item $\nabla f(x,y) = \lambda \nabla g(x,y)$
	\item $g(x,y) = C$
\end{itemize}

\section{Multiple Integrals}

Fubini's Theorem: If $f(x,y)$ is continuous on the domain $D$:\\
\tab $\iint_D f(x,y)\,dA = \int_{c}^{d} \hspace{-1mm} \int_{a}^{b}f(x,y)\,dx\,dy = \int_{a}^{b} \hspace{-1mm} \int_{c}^{d}f(x,y)\,dy\,dx$

Area of the domain $D$:\\
\tab $ A = \iint_D\,dA$

Average value of a $f(x,y)$ over domain $D$:\\
\tab $f_{\text{avg}} = \frac{1}{A}\iint_D f(x,y)\,dA$

Mass of a lamina $D$ with density $\rho (x,y)$:\\
\tab $m = \iint_D \rho (x,y)\,dA$

Moment of a mass about the $x$-axis:\\
\tab $M_x = \iint_D x\rho (x,y) \,dA$

Center of mass of a lamina:\\
\tab $\bar{x} = M_x/m \hspace{1cm} \bar{y} = M_y/m \hspace{1cm} \bar{z} = M_z/m$

Surface area of $f(x,y)$:\\
\tab $\int_{x_0}^{x_1}\int_{y_0}^{y_1} \sqrt{f_x^2+f_y^2+1}\,dy\,dx.$

\subsubsection{Cylindrical Coordinates}
\tab $x=r\cos\theta \hspace{1cm} y=r\sin\theta \hspace{1cm} z = z$\\
\tab $dA = r\,dr\,d\theta \hspace{1cm} dV = r\,dr\,dz\,d\theta$\\
\tab $x^2 + y^2 = r^2$
\subsubsection{Spherical Coordinates}
\tab $x= \rho\sin\phi\cos\theta$\\
\tab $y= \rho\sin\phi\sin\theta$\\
\tab $z= \rho\cos\phi$\\
\tab $x^2+y^2+z^2 = \rho^2$\\
\tab $x^2+y^2 = \rho^2\sin^2\phi$\\
\tab $dV = \rho^2\sin\phi\,d\rho\,d\phi\,d\theta$\\

\subsubsection{Change of Variables}
Suppose that $f(x,y)$ is continuous on $R$ and that $R$ and $S$ are type I or type II plane regions. Suppose also that $T$ is one-to-one, except perhaps on the boundary of $S$:\\
$\iint_R F(x,y) \, dV = 
\iint_S F(x(u,v),y(u,v)) 
\left|{\partial(x,y)\over\partial(u,v)}\right| \,du\,dv$
$\small\text{3D case:}\iiint_R F(x,y,z) \, dV =$\\
	\hspace{-0.5cm}\tab $\iiint_S F(x(u,v,w),y(u,v,w),z(u,v,w)) 
	\left|{\partial(x,y,z)\over\partial(u,v,w)}\right| \,du\,dv\,dw$\\
Jacobian:\\
\tab $\ds \left|{\partial(x,y)\over\partial(u,v)}\right| = \det\mqty[ x_u& x_v\\ y_u& y_v ]$\\
\tab $\ds \left|{\partial(x,y,z)\over\partial(u,v,w)}\right| = \det\mqty[ 
	x_u& x_v&x_w\\ 
	y_u& y_v&y_w \\
	z_u&z_v&z_w]$

\section{Vector Calculus}
A vector field $\vb{F}$ is conservative if there is a scalar function $f$ such that: $\vb{F} = \nabla f$\\
\tab $\text{curl }\vb{F} = \curl{\vb{F}}$\\
\tab $\text{div }\vb{F} = \div{\vb{F}}$\\
\tab $\nabla\times(\grad{f}) = \vb{0}$
		
If $\div{\vb{F}} \neq 0$, then $\vb{F}$ cannot be a curl of another vector field.

\subsubsection{Line Integrals}
\tab $\int_C f(x,y,z)\,ds = \int_{a}^{b} f(\vb{r}(t))\,|\vb{r}'(t)|\,dt$\\
where $\vb{r}(t)$ is the parametrization of $C$.

\tab $\int_C \vb{F}\cdot d\vb{r} = \int_{a}^{b} \vb{F(\vb{r}(t))}\cdot \vb{r}'(t)\,dt = \int_{a}^{b} \vb{F}\cdot\vb{T}\,ds$\\
where $\vb{T}$ is the unit tangential vector $\frac{\vb{r}'(t)}{|\vb{r}'(t)|}$

Fundamental Theorem of Line Integrals:\\
\tab $\int_C \grad{f} \cdot d\vb{r} = f(\vb{r}(b)) - f(\vb{r}(a))$

\subsubsection{Independence of Path \& Conservativeness}
\begin{itemize}[leftmargin=0.5cm]
	\vspace{-0.5em}
	\setlength\itemsep{-0.3em}
\item Let $\vb{F}$ be a continuous vector field on the domain $D$. We have independence of path in $D$ if:\\
\tab $\int_{C_1} \vb{F}\cdot d\vb{r} = \int_{C_2} \vb{F}\cdot d\vb{r}$\\
\item If $\vb{F}$ is conservative, then $\int_{C} \vb{F}\cdot d\vb{r}$ is independent of path.

\item If Clairaut's Theorem fails, then $\vb{F}$ is not conservative.

\item If $D$ is an open (boundary points are not on the domain) and connected region and $\vb{F}$ is a continuous vector field of $D$, then if $\int_{C_1} \vb{F}\cdot d\vb{r}$ is independent of path in $D$, then $\vb{F}$ is conservative.

\item Green's Theorem evaluates to 0 for any conservative vector field.

\item $\vb{F}$ is conservative if and only if $\nabla \times \vb{F} = 0$
\end{itemize}

\subsubsection{Green's Theorem}
Let $C$ be a simple piecewise smooth curve that bounds a region $D$ in the plane. If $P(x,y)$ and $Q(x,y)$ have continuous partials in an open region containing $D$, then\\
\tab $\int_C P\,dx +Q\,dy = \iint_D \pdv{Q}{x}
	-\pdv{P}{y} \,dA$

If $\vb{F}$ is a vector field with third component 0 defined on a domain $D$ enclosed by boundary $C$ then\\
\tab $\oint_{C} \vb{F}\cdot d\vb{r} = \iint_D (\curl{\vb{F}})\cdot\vb{k}\,dA.$

Similarly, if $C$ is defined by $\vb{r}(t) = \left<x(t),y(t)\right>$\\
\tab $\oint_{C} \vb{F}\cdot \vb{n}\,ds = \iint_D \div{\vb{F}}\,dA$

\subsubsection{Parametric Surfaces}
Plane through $\vb{r}_0$ parallel to the non-parallel vectors $\vb{v}_1$, $\vb{v}_2$:\\
\tab $\vb{r}(s,t) = \vb{r}_0 + s\vb{v}_1 + t\vb{v}_2$

Graph of a function $z = f(x,y)$:\\
\tab $\vb{r}(x,y) = \left<x,\,y,\,f(x,y)\right>$

Cylinder about the x-axis:\\
\tab $\vb{r}(x,\theta) = \left<x,\,\cos\theta,\,\sin\theta\right>$

A cone given by $z = a\sqrt{x^2 + y^2}$:\\
\tab $\vb{r}(r,\theta) = \left<r\cos\theta,\,r\sin\theta,\,ar\right>$

An ellipsoid given by $\frac{x^2}{a^2} + \frac{y^2}{b^2} + \frac{z^2}{c^2} = 1$\\
\tab $\vb{r}(\phi,\theta) = \left<a\sin\phi\cos\theta,\,b\sin\phi\sin\theta,\,c\cos\theta\right>$

A smooth parametric surface given by the equation $\vb{r}(u,v) = \left<x(u,v),\,y(u,v),\,z(u,v)\right>$ in a domain $D$ and $S$ is covered once throughout the parameter domain $D$, then the area of $S$ is:\\
\tab $A(S) = \iint_D |\vb{r}_u \times \vb{r}_v|\,dA$

Surface area of a graph of a function:\\
\tab $A(S) = \iint_D \sqrt{1 + (f_x)^2 + (f_y)^2}\,dA$


\subsubsection{Surface Integrals}

$\iint_S f(x,y,z)\,d\vb{S} = \iint_D f(\vb{r}(u,v))\, |\vb{r}_u \times \vb{r}_v|\,dA$\\
\vspace{0.2cm}
Graph of a function $z = g(x,y)$:\\
$\iint_S f(x,y,z)\,d\vb{S} = \iint_D f(x,y,g(x,y)) \sqrt{1 + (g_x)^2 + (g_y)^2}\,dA$\\
\vspace{0.2cm}
A surface $S$ given by a graph $z = g(x,y)$:\\
\tab $\vb{F} \cdot (\vb{r}_x \times \vb{r}_y) = \left< P,Q,R \right> \cdot \left<-\pdv{g}{x},-\pdv{g}{y},1\right> \text{  so}$\\
\tab $\iint_S \vb{F} \cdot \,d\vb{S} = \iint_D (-P\pdv{g}{x} - Q\pdv{g}{y} + R)\, dA$\\
Vector field:\\
$\iint_S f(x,y,z)\,d\vb{S} = \iint_S \vb{F}\cdot\vb{n}\,dS = \iint_D \vb{F} \cdot (\vb{r}_u \times \vb{r}_v)\, dA$

\subsubsection{Stokes' Theorem}
Let $S$ be an oriented piecewise-smooth surface that is	bounded by a simple, closed, piecewise-smooth boundary curve $C$ with positive orientation. Let $\vb{F}$ be a vector field whose components have continuous partial derivatives on an open region in $\mathbb{R}^3$ that contains $S$. Then\\
$\oint_{C} \vb{F}\cdot d\vb{r}
=\iint_D(\nabla\times \vb{F})\,d\vb{S} = \iint_D(\nabla\times \vb{F})\cdot \vb{n}\,dA$

\subsubsection{Divergence Theorem}
Let $E$ be a solid region in $\mathbb{R}^3$ with piecewise smooth boundary surface $S$ (given the outward orientation). Let $\vb{F}$ be a vector field with continuous partial derivatives on a region containing $E$. Then\\
\tab $\iint_D \vb{F}\cdot\,d\vb{S} = \iint_D \vb{F}\cdot\vb{n}\,dS=\iiint_E \nabla\cdot\vb{F}\,dV.$\\
The flux of $\vb{F}$ across the boundary surface of $E$ is equal to the triple integral of the divergence of $\vb{F}$ over $E$.\\
For cases where it is too difficult to parameterize a surface, the surface is not closed, and we cannot use Stokes' Theorem, we can close the surface and apply the Divergence Theorem. (Later subtract the contribution from the closed surface)

\subsubsection{Trigonometric Identities}
\begin{itemize}[leftmargin=0.5cm]
	\vspace{-0.5em}
	\setlength\itemsep{-0.3em}
	\item $\ds \sin^2 x  = \frac{1 - \cos(2x)}{2}$
	\item $\ds \cos^2 x  = \frac{1 + \cos(2x)}{2}$
	\item $\sin(2x) = 2\sin x\cos x$
	\item $\cos(2x)  = \cos^2 x - \sin^2 x = 2 \cos^2 x - 1 = 1 - 2 \sin^2 x$
\end{itemize}

% Footer content
\rule{0.3\linewidth}{0.25pt}
\scriptsize\\
Updated \today\\
\href{https://github.com/DonneyF/formula-sheets}{https://github.com/DonneyF/formula-sheets}

\end{multicols}
\end{document}