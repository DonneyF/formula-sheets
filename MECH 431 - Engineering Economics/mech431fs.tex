\documentclass[12pt,landscape]{article}
\usepackage{multicol}
\usepackage{calc}
\usepackage{ifthen}
\usepackage[landscape]{geometry}
\usepackage{amsmath,amsthm,amsfonts,amssymb}
\usepackage{color,graphicx,overpic}
\usepackage{hyperref}
\usepackage{enumitem}
\usepackage{upgreek}
\usepackage{physics}
\usepackage{newtxtext,newtxmath}

% This sets page margins to .5 inch if using letter paper, and to 1cm
% if using A4 paper. (This probably isn't strictly necessary.)
% If using another size paper, use default 1cm margins.
\ifthenelse{\lengthtest { \paperwidth = 11in}}
	{ \geometry{top=.5in,left=.5in,right=.5in,bottom=.5in} }
	{\ifthenelse{ \lengthtest{ \paperwidth = 297mm}}
		{\geometry{top=1cm,left=1cm,right=1cm,bottom=1cm} }
		{\geometry{top=1cm,left=1cm,right=1cm,bottom=1cm} }
	}

% Turn off header and footer
\pagestyle{empty}
 

% Redefine section commands to use less space
\makeatletter
\renewcommand{\section}{\@startsection{section}{1}{0mm}%
                                {-1ex plus -.5ex minus -.2ex}%
                                {0.5ex plus .2ex}%x
                                {\normalfont\normalsize\bfseries}}
\renewcommand{\subsection}{\@startsection{subsection}{2}{0mm}%
                                {-1explus -.5ex minus -.2ex}%
                                {0.5ex plus .2ex}%
                                {\normalfont\small\bfseries}}
\renewcommand{\subsubsection}{\@startsection{subsubsection}{3}{0mm}%
                                {-1ex plus -.5ex minus -.2ex}%
                                {1ex plus .2ex}%
                                {\normalfont\footnotessize\bfseries}}
\makeatother

% Define BibTeX command
\def\BibTeX{{\rm B\kern-.05em{\sc i\kern-.025em b}\kern-.08em
    T\kern-.1667em\lower.7ex\hbox{E}\kern-.125emX}}

% Don't print section numbers
\setcounter{secnumdepth}{0}


\setlength{\parindent}{0pt}
\setlength{\parskip}{1pt plus 0.5ex}

\newcommand{\tab}{\hspace{.02\textwidth}}
\newcommand{\ds}{\displaystyle}
\newcommand{\ap}{a_+}
\newcommand{\am}{a_-}

\renewcommand{\dv}[2]{\frac{d#1}{d#2}}

% Redefine some commands for newtxmath boldness
\renewcommand{\grad}{\nabla}
\renewcommand{\curl}[1]{\nabla\times#1}
\renewcommand{\div}[1]{\nabla\cdot#1}
\renewcommand{\cross}{\times}

% -----------------------------------------------------------------------

\begin{document}

\raggedright
\footnotesize
\begin{multicols*}{3}

% multicol parameters
% These lengths are set only within the two main columns
%\setlength{\columnseprule}{0.25pt}
\setlength{\premulticols}{1pt}
\setlength{\postmulticols}{1pt}
\setlength{\multicolsep}{1pt}
\setlength{\columnsep}{2pt}

\begin{center}
	\Large{\underline{MECH 431 Formula Sheet}}
\end{center}

\section{Engineering Costs and Cost Estimating}
Engineering Costs:
\begin{itemize}
	\itemsep 0em
	\item Fixed - Constant regardless of output activity
	\item Variable - Depends on output activity
	\item Marginal - Variable cost of one more unit
	\item Average - Total cost divided by number of units
	\item Total - Total Fixed + Total Variable
	\item Sunk - Money already spend, result of a past decision
	\item Opportunity - Next best benefit forgone
	\item Recurring - Repeating expense that is known, anticipated
	\item Non-Recurring - One-of-a-kind, Irregular
	\item Incremental - Cost differences between alternatives
	\item Cash - Costs associated with cash transactions
	\item Book - Cost effects from past decisions
	\item Life-Cycle - Costs over various phases of a product's life
\end{itemize}

Estimating Models:
\begin{itemize}
	\itemsep 0em
	\item Per-Unit - Per-unit factor
	\item Segmenting - Divide \& conquer
	\item Cost Indices - Historical changes based on ratio
	$$\frac{\text{Cost at time A}}{\text{Cost at time B}} = \frac{\text{Index value at time A}}{\text{Index value at time B}}$$
	\item Power-Sizing - Accounts for Economies of Scale
	$$\frac{\text{Cost of equipment A}}{\text{Cost of equipment B}} = \left(\frac{\text{Capacity of equipment A}}{\text{Capacity of equipment B}}\right)^x$$
	\item Learning Curve - Relationship between repetition and performance
	$$T_N = T_i \times N^b$$
	$$b = \log_2(\text{learning curve expressed as a decimal})$$
	for N completed units.
\end{itemize}

\section{Interest and Equivalence}
Simple Interest:\\
\tab $F = P(1 + in)$

Single-Payment Compound Interest:\\
\tab $F = P(1 + i)^n$

Single-Payment Present Worth:\\
\tab $P = F(1 + i)^{-n}$

Effective Annual Interest Rate for a nominal interest rate ($r$) and $m$ compounding subperiods:\\
\tab $\ds i_a = \left(1+ \frac{r}{m}\right)^m -1$ \qquad $i_a = (1 + i)^m-1$

Uniform Series Compound Amount/Sinking Fund:\\
\tab $\ds F = A\left[\frac{(1+i)^n-1}{i}\right]$ \qquad $\ds A = F\left[\frac{i}{(1+i)^n-1}\right]$

Uniform Series Capital Recovery/Present Worth:\\
\tab $\ds A = P\left[\frac{i(1+i)^n}{(1+i)^n-1}\right]$ \qquad $\ds P = A\left[\frac{(1+i)^n-1}{i(1+i)^n}\right]$

Arithmetic Gradient Present Worth:\\
\tab $\ds P = G\left[\frac{(1+i)^n-in-1}{i^2(1+i)^n}\right]$

Arithmetic Gradient Uniform Series:\\
\tab $\ds A = G\left[\frac{(1+i)^n-in-1}{i(1+i)^n-i}\right]$

Geometric Gradient Present Worth:\\
\tab $\ds P = A_1\left[\frac{1-(1+g)^n(1+i)^{-n}}{i-g}\right]$ \quad for $i\neq g$\\
\tab $\ds P = A_1n(1+i)^{-1}$ \hspace{2.27cm} for $i = g$


% Footer content
\rule{0.3\linewidth}{0.25pt}
\scriptsize\\
Updated \today\\
\href{https://github.com/DonneyF/formula-sheets}{https://github.com/DonneyF/formula-sheets}
\end{multicols*}
\end{document}
