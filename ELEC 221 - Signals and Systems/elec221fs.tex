\documentclass[12pt,landscape]{article}
\usepackage{multicol}
\usepackage{calc}
\usepackage{ifthen}
\usepackage[landscape]{geometry}
\usepackage{amsmath,amsthm,amsfonts,amssymb}
\usepackage{hyperref}
\usepackage{enumitem}
\usepackage{upgreek}
\usepackage{physics}
\usepackage{mathtools}
\usepackage{newtxtext,newtxmath}
\usepackage{booktabs}

% This sets page margins to .5 inch if using letter paper, and to 1cm
% if using A4 paper. (This probably isn't strictly necessary.)
% If using another size paper, use default 1cm margins.
\ifthenelse{\lengthtest { \paperwidth = 11in}}
	{ \geometry{top=.5in,left=.5in,right=.5in,bottom=.5in} }
	{\ifthenelse{ \lengthtest{ \paperwidth = 297mm}}
		{\geometry{top=1cm,left=1cm,right=1cm,bottom=1cm} }
		{\geometry{top=1cm,left=1cm,right=1cm,bottom=1cm} }
	}

% Turn off header and footer
\pagestyle{empty}
 

% Redefine section commands to use less space
\makeatletter
\renewcommand{\section}{\@startsection{section}{1}{0mm}%
                                {-1ex plus -.5ex minus -.2ex}%
                                {0.5ex plus .2ex}%x
                                {\normalfont\normalsize\bfseries}}
\renewcommand{\subsection}{\@startsection{subsection}{2}{0mm}%
                                {-1explus -.5ex minus -.2ex}%
                                {0.5ex plus .2ex}%
                                {\normalfont\small\bfseries}}
\renewcommand{\subsubsection}{\@startsection{subsubsection}{3}{0mm}%
                                {-1ex plus -.5ex minus -.2ex}%
                                {1ex plus .2ex}%
                                {\normalfont\footnotessize\bfseries}}
\makeatother

% Don't print section numbers
\setcounter{secnumdepth}{0}

\setlength{\parindent}{0pt}
\setlength{\parskip}{1pt plus 0.5ex}

\newcommand{\tab}{\hspace{.02\textwidth}}
\newcommand{\ds}{\displaystyle}

\renewcommand{\dv}[2]{\frac{d#1}{d#2}}
\DeclareMathOperator{\sinc}{sinc}

% -----------------------------------------------------------------------

\begin{document}

\raggedright
\footnotesize
\begin{multicols*}{3}


% multicol parameters
% These lengths are set only within the two main columns
%\setlength{\columnseprule}{0.25pt}
\setlength{\premulticols}{1pt}
\setlength{\postmulticols}{1pt}
\setlength{\multicolsep}{1pt}
\setlength{\columnsep}{2pt}

\begin{center}
	\Large{\underline{ELEC 221 Formula Sheet}}
\end{center}

\section{Continuous Time Signals}

Even and Odd Components:\\
\tab $x_e(t) = \frac{1}{2}[x(t) + x(-t)]$\\
\tab $x_o(t) = \frac{1}{2}[x(t) - x(-t)]$

A signal is periodic with fundamental period $T_0$ if\\
\tab $x(t + kT_0) = x(t) \quad \forall t \in (-\infty, \infty)$

Energy:\\
\tab $E = \int_{-\infty}^{\infty} \abs{x(t)}^2\,dt$

Power:\\
\tab $ P = \lim_{T \rightarrow \infty} \frac{1}{2T} \int_{-T}^{T}\abs{x(t)}^2\,dt$

Causality:\\
\begin{itemize}
	\itemsep0em
	\item Causal: $x(t) = 0$ for $t < 0$.
	\item Anti-Causal: $x(t) = 0$ for $t \geq 0$.
	\item A-Causal or Non-Causal: Both of the above.
\end{itemize}

\section{Continuous Time Systems}
Dynamic systems have memory. Active systems can deliver energy to the outside world.
\\
Linearity:\\
\tab $\mathcal{S}[\alpha x(t) + \beta y(t)] = \alpha\mathcal{S}[x(t)] + \beta\mathcal{S}[y(t)]$

Time Invariance:\\
\tab If $\mathcal{S}[x(t)] = y(t)$ then $\mathcal{S}[x(t \pm \tau)] = y(t \pm \tau)$.

Zero-State Response:\\
\tab Due to the input as the initial conditions are zero.

Zero-Input Response:\\
\tab Due to the initial conditions as the input is zero.

Convolution:\\
\tab $y(t) = \int_{-\infty}^{\infty}x(\tau)h(t-\tau)\,d\tau$

Causality:\\
\tab A continuous time system $\mathcal{S}$ is causal if whenever $x(t) = 0$ and there are no initial conditions, $y(t) = 0$ and the output $y(t)$ does not depend on future inputs.
\\~\\
Bounded-Input Bounded-Output Stability:\\
\tab If an input $x(t)$ bounded then the output of an BIBO system is also bounded.\\
\tab $\int_{-\infty}^{\infty} \abs{h(t)}\,dt < \infty$

\section{Laplace Transform}
$s = \sigma + j\omega$\\

Eigenfunction Property:\\
\tab $\mathcal{S}[e^{s_0t}] = H(s_0)e^{s_0t}$

One Sided Laplace Transform:\\
\tab $F(s) = \mathcal{L}[f(t)u(t)] = \int_{0^-}^{\infty}f(t)e^{-st}\,dt$

Region of Convergence:
\begin{itemize}
	\itemsep0em
	\item Finite support $\rightarrow$ Entire s-plane.
	\item Causal function $\rightarrow$ $\sigma > \max(\sigma_i)$,  $-\infty < \omega < \infty$.
	\item Anti-causal $\rightarrow$ $\sigma < \min(\sigma_i)$,  $-\infty < \omega < \infty$.
	\item Non-causal $\rightarrow$ $\mathcal{R} = \mathcal{R}_\text{causal} \cap \mathcal{R}_\text{anti-causal}$.
\end{itemize}

Initial Value Theorem:\\
\tab $\ds f(0+) \Leftrightarrow \lim_{s\rightarrow\infty} sF(s)$

Final Value Theorem:\\
\tab $\ds \lim_{t\rightarrow\infty} f(t) \Leftrightarrow \lim_{s\rightarrow 0} sF(s)$

Bounded-Input Bounded-Output Stability:\\
\tab If the region of convergence contains the $j\omega$-axis, then the system is BIBO stable.

\section{Fourier Series}
\tab Fourier analysis in the steady state.

Eigenfunction Property:\\
\tab $\mathcal{S}[e^{j\omega_0t}] = H(j\omega_0)e^{j\omega_0t}$\\
\tab $x(t) = \sum_k X_k e^{j\omega_kt} \implies y(t) = \sum_k X_k H(j\omega_k)e^{j\omega_kt}$\\\vspace{1mm}\hspace{2.5cm}$= \sum_k X_k \abs{H(j\omega_k)}e^{j(\omega_kt + \angle H(\omega_k))}$

Fourier Series Coefficients (for any $t_0$):\\
\tab $X_k = \frac{1}{T_0}\int_{t_0}^{t_0 + T_0}x(t)e^{-jk\omega_0t}\,dt$

Parseval's Power Relation (for any $t_0$):\\
\tab $ P = \frac{1}{T_0} \int_{t_0}^{t_0 + T_0}\abs{x(t)}^2\,dt = \sum_{k = -\infty}^{\infty}\abs{X_k}^2$

Symmetry of Line Spectra:\\
\tab $\abs{X_k} = \abs{X_{-k}}$\\
\tab $\angle X_k = -\angle X_{-k}$

Trigonometric Fourier Series:\\
\tab $x(t) = X_0 + 2\sum_{k = 1}^{\infty}\abs{X_k}\cos(k\omega t + \Theta_k)$\\
\tab $x(t) = c_0 + 2\sum_{k = 1}^{\infty}[c_k\cos(k\omega_0 t) + d_k\sin(k\omega_0 t)]$\\
\tab $c_k = \frac{1}{T_0} \int_{t_0}^{t_0 + T_0}x(t)\cos(k\omega_0 t)\,dt$ \quad $k$ = 0,1,2$\ldots$\\
\tab $d_k = \frac{1}{T_0} \int_{t_0}^{t_0 + T_0}x(t)\sin(k\omega_0 t)\,dt$\quad $k$ = 1,2,3$\ldots$\\
\tab $\Theta_k = -\arctan(d_k/c_k)$
\\~\\
Fourier Coefficients from Laplace Transform:\\
\tab If $x_1(t)$ is a single period of $x(t)$, then\\
\tab $X_k = \frac{1}{T_0}\mathcal{L}[x_1(t)] \big\rvert_{s = jk\omega_0}$

Response of LTI Systems to Periodic Signals:\\
\tab If the input to an LTI system has Fourier Series $x(t) = X_0 + 2\sum_{k=1}^{\infty}\abs{X_k}\cos(k\omega t + \angle X_k)$, then the steady state response is $y(t) = X_0\abs{H(j0)} + 2\sum_{k=1}^{\infty}\abs{X_k}\abs{H(jk\omega_0)}\cos(k\omega_0 t + \angle X_k + \angle H(jk\omega_0))$

\section{Fourier Transform}
\tab $X(\omega) = \int_{-\infty}^{\infty}x(t)e^{-j\omega t}\,dt$\\
\tab $x(t) = \frac{1}{2\pi}\int_{-\infty}^{\infty}X(\omega)e^{j\omega t}\,d\omega$

Fourier Transform from Laplace Transform (if $X(s)$ contains the $j\omega$-axis):\\
\tab $\mathcal{F}[x(t)] = \mathcal{L}[x(t)]\big\rvert_{s = j\omega} = X(s)\big\rvert_{s = j\omega}$

Duality:\\
\tab $\ds X(t) \xLeftrightarrow{\mathcal{F}} 2\pi x(-\omega)$

Fourier Transform of Periodic Signals:\\
\tab $\sum_k X_k e^{jk\omega_0 t} \xLeftrightarrow{\mathcal{F}} \sum_k 2\pi_k \delta(\omega - k\omega_0)$

Parseval's Energy Relation:\\
\tab $E = \int_{-\infty}^{\infty} \abs{x(t)}^2\,dt = \frac{1}{2\pi}\abs{X(\omega)}^2\,d\omega$

Symmetry of Spectral Representations:\\
\tab $\abs{X(\omega)} = \abs{X(-\omega)}$\\
\tab $\Re[X(\omega)] = \Re[X(-\omega)]$\\
\tab $\angle X(\omega) = -\angle X(-\omega)$\\
\tab $\Im[X(\omega)] = -\Im[X(-\omega)]$

\section{Sampling Theory}
\tab $x_s(t) = x(nT_s) = x(t)\rvert_{t = nT_s} = x(t)\sum_n \delta(t - nT_s)$\\
\tab $X_s(\omega) = \frac{1}{T_s} \sum_{k=-\infty}^{\infty}X(\omega - k\omega_s)$

Nyquist-Shannon Sampling Rate:\\
\tab $\omega_s = \frac{2\pi}{T_s} \geq 2\omega_\text{max}$\\
\tab Aliasing occurs if $\omega_s < 2\omega_\text{max}$.

Reconstruction $X(\omega)=X_s(\omega)H_\text{lp}(\omega)$:
\begin{equation*}
H_{\text{lp}}(\omega) =
\begin{cases}
T_s & -\frac{\omega_s}{2} < \omega < \frac{\omega_s}{2}\\
0 & \text{otherwise}
\end{cases}
\end{equation*}

Signal Reconstruction from Sinc Interpolation:\\
\tab $x_r(t) = \sum_{n=-\infty}^{\infty} x(nT_s) \frac{\sin(\pi(t-nT_s))/T_s}{\pi(t-nT_s)/T_s}$
\end{multicols*}

\newpage

\begin{minipage}[t]{0.30\textwidth}
\section{Discrete Time Signals}
\tab Define $x[n] = x(nT_0)$.\\
\tab $x[n] = \sum_{k=-\infty}^{\infty}x[k]\delta[n-k]$

Periodicity:\\
\tab $x[n + kN] = x[n]$ \qquad $\forall k\in \mathbb{Z}$

When sampling an analog sinusoid of fundamental period $T_0$, we obtain a periodic discrete sinusoid provided that $m$, $N$ not divisible by each-other:\\
\tab $T_s/T_0 = m/N$

Aliasing occurs if:\\
\tab $T_s > T_0/2$

Energy:\\
\tab $E = \sum_{n = -\infty}^{\infty}\abs{x[n]}^2$

Power:\\
\tab $ P_x = \lim_{N \rightarrow \infty}\frac{1}{2N + 1}\sum_{n= -N}^{N}\abs{x[n]}^2$

Convolutional Sum:\\\
\tab $y[n] = \sum_{k = -\infty}^{\infty} x[k]h[n-k]$

Bounded-Input Bounded-Output Stability:\\
\tab $\sum_k \abs{h[n]}^2 < \infty$

\section{Z-Transform}
\tab $\mathcal{Z}[x(nT_s)] = \mathcal{L}[x_s(t)]\rvert_{z=e^{sT_s}} = \sum_n x(nT_s)z^{-n}$

Convergence:\\
\tab $|X(z)| = \sum_n \abs{x[n]}r^{-n} < \infty$
% Footer content
\rule{0.3\linewidth}{0.25pt}
\scriptsize\\
Updated \today\\
\href{https://github.com/DonneyF/formula-sheets}{https://github.com/DonneyF/formula-sheets}
\end{minipage}%
\begin{minipage}[t]{0.43\textwidth}
\subsection{Fourier Transform Properties}
\begin{tabular}{lll}  
	\toprule
	   & Time Domain & Frequency Domain\\
	\midrule
	Expansion/Contraction & $x(\alpha t)$, $\alpha \neq 0$ & $\frac{1}{\abs{\alpha}} X(\frac{\omega}{\alpha})$ \\
	Duality & $X(t)$ & $2\pi x(-\omega)$\\
	Differentiation ($t$) & $\ds \frac{d^nx}{dt^2}$& $(j\omega)^n X(\omega)$\\
	Differentiation ($\omega$) & $-jtx(t)$ & $\ds \dv{X(\omega)}{\omega}$\\
	Integration & $\ds \int_{-\infty}^{t} x(\tau)\,d\uptau$ & $\ds \frac{X(\omega)}{j\omega} + \pi X(0)\delta(\omega)$\\
	Time shifting & $x(t - \alpha)$ & $e^{-j\alpha\omega}X(\omega)$\\
	Frequency shifting &  $e^{j\omega_0 t}x(t)$ & $X(\omega-\omega_0)$\\
	Modulation & $x(t)\cos(\omega_c t)$ & $0.5[X(\omega - \omega_c) + X(\omega + \omega_c)]$\\
	\bottomrule
\end{tabular}
\end{minipage}%
\begin{minipage}[t]{0.20\textwidth}
\subsection{Fourier Transform Pairs}
\begin{tabular}{ll}  
	\toprule
	$f(t)$ & $F(\omega)$\\
	\midrule
	$\delta(t)$ & 1\\
	$\delta(t - \tau)$ & $e^{-j\omega\tau}$\\
	$u(t)$ & $1/j\omega + \pi\delta(\omega)$\\
	$u(t)$ & $-1/j\omega + \pi\delta(\omega)$\\
	$\text{sign}(t)$ & $2/j\omega$\\
	$e^{-a\abs{t}}$, $a > 0$ & $\frac{2a}{a^2 + \omega^2}$\\
	$\cos(\omega_0 t)$ & $\pi[\delta(\omega - \omega_0)] + \delta(\omega + \omega_0)$\\
	$\sin(\omega_0 t)$ & $-j\pi[\delta(\omega - \omega_0)] - \delta(\omega + \omega_0)$\\
	$\frac{\sin(\omega t)}{\pi t}$ &  $u(\omega + \omega_0) - u(\omega - \omega_0)$\\
	$u(\omega + \tau) - u(\omega - \tau)$ & $2\tau \frac{\sin(\omega\tau)}{\omega\tau}$\\
	\bottomrule
\end{tabular}
\end{minipage}


\end{document}
