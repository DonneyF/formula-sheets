\documentclass[12pt,landscape]{article}
\usepackage{multicol}
\usepackage{calc}
\usepackage{ifthen}
\usepackage[landscape]{geometry}
\usepackage{amsmath,amsthm,amsfonts,amssymb}
\usepackage{color,graphicx,overpic}
\usepackage{hyperref}
\usepackage{enumitem}
\usepackage{upgreek}
\usepackage{physics}
\usepackage{mathtools}
\usepackage{newtxtext,newtxmath}

% This sets page margins to .5 inch if using letter paper, and to 1cm
% if using A4 paper. (This probably isn't strictly necessary.)
% If using another size paper, use default 1cm margins.
\ifthenelse{\lengthtest { \paperwidth = 11in}}
	{ \geometry{top=.5in,left=.5in,right=.5in,bottom=.5in} }
	{\ifthenelse{ \lengthtest{ \paperwidth = 297mm}}
		{\geometry{top=1cm,left=1cm,right=1cm,bottom=1cm} }
		{\geometry{top=1cm,left=1cm,right=1cm,bottom=1cm} }
	}

% Turn off header and footer
\pagestyle{empty}
 

% Redefine section commands to use less space
\makeatletter
\renewcommand{\section}{\@startsection{section}{1}{0mm}%
                                {-1ex plus -.5ex minus -.2ex}%
                                {0.5ex plus .2ex}%x
                                {\normalfont\normalsize\bfseries}}
\renewcommand{\subsection}{\@startsection{subsection}{2}{0mm}%
                                {-1explus -.5ex minus -.2ex}%
                                {0.5ex plus .2ex}%
                                {\normalfont\small\bfseries}}
\renewcommand{\subsubsection}{\@startsection{subsubsection}{3}{0mm}%
                                {-1ex plus -.5ex minus -.2ex}%
                                {1ex plus .2ex}%
                                {\normalfont\footnotessize\bfseries}}
\makeatother

% Define BibTeX command
\def\BibTeX{{\rm B\kern-.05em{\sc i\kern-.025em b}\kern-.08em
    T\kern-.1667em\lower.7ex\hbox{E}\kern-.125emX}}

% Don't print section numbers
\setcounter{secnumdepth}{0}


\setlength{\parindent}{0pt}
\setlength{\parskip}{1pt plus 0.5ex}

\newcommand{\tab}{\hspace{.02\textwidth}}
\newcommand{\ds}{\displaystyle}

\def\rcurs{{\mbox{$\resizebox{.09in}{.08in}{\includegraphics[trim= 1em 0 14em 0,clip]{ScriptR}}$}}}
\def\brcurs{{\mbox{$\resizebox{.09in}{.08in}{\includegraphics[trim= 1em 0 14em 0,clip]{BoldR}}$}}}

\renewcommand{\dv}[2]{\frac{d#1}{d#2}}

\DeclareMathOperator{\sinc}{sinc}

% -----------------------------------------------------------------------

\begin{document}

\raggedright
\footnotesize
\begin{multicols*}{3}


% multicol parameters
% These lengths are set only within the two main columns
%\setlength{\columnseprule}{0.25pt}
\setlength{\premulticols}{1pt}
\setlength{\postmulticols}{1pt}
\setlength{\multicolsep}{1pt}
\setlength{\columnsep}{2pt}

\begin{center}
	\Large{\underline{ELEC 221 Formula Sheet}}
\end{center}

\section{Continuous Time Signals}

Even and Odd Components:\\
\tab $x_e(t) = \frac{1}{2}[x(t) + x(-t)]$\\
\tab $x_o(t) = \frac{1}{2}[x(t) - x(-t)]$

A signal is periodic with fundamental period $T_0$ if\\
\tab $x(t + kT_0) = x(t) \quad \forall t \in (-\infty, \infty)$

Energy:\\
\tab $\ds E = \int_{-\infty}^{\infty} \abs{x(t)}^2\,dt$

Power:\\
\tab $\ds P = \lim_{T \rightarrow \infty} \frac{1}{2T} \int_{-T}^{T}\abs{x(t)}^2\,dt$

Causality:\\
\begin{itemize}
	\itemsep0em
	\item Causal: $x(t) = 0$ for $t < 0$.
	\item Anti-Causal: $x(t) = 0$ for $t \geq 0$.
	\item A-Causal or Non-Causal: Both of the above.
\end{itemize}

\section{Continuous Time Systems}
Dynamic systems have memory. Active systems can deliver energy to the outside world.
\\
Linearity:\\
\tab $\mathcal{S}[\alpha x(t) + \beta y(t)] = \alpha\mathcal{S}[x(t)] + \beta\mathcal{S}[y(t)]$

Time Invariance:\\
\tab If $\mathcal{S}[x(t)] = y(t)$ then $\mathcal{S}[x(t \pm \tau)] = y(t \pm \tau)$.

Zero-State Response:\\
\tab Due to the input as the initial conditions are zero.

Zero-Input Response:\\
\tab Due to the initial conditions as the input is zero.

Convolution Integral:\\
\tab $\ds [x * y](t) = \int_{-\infty}^{\infty}x(\tau)y(t-\tau)\,d\tau = \int_{-\infty}^{\infty}x(t - \tau)y(\tau)\,d\tau$

Total Response from Impulse Response:\\
\tab $\ds y(t) = \int_{-\infty}^{\infty}x(\tau)h(t-\tau)\,d\tau$

Causality:\\
\tab A continuous time system $\mathcal{S}$ is causal if whenever $x(t) = 0$ and there are no initial conditions, $y(t) = 0$ and the output $y(t)$ does not depend on future inputs.
\\~\\
Bounded-Input Bounded-Output Stability:\\
\tab If an input $x(t)$ bounded then the output of an BIBO system is also bounded.\\
\tab $\ds \int_{-\infty}^{\infty} \abs{h(t)}\,dt < \infty$

\section{Laplace Transform}
$s = \sigma + j\omega$\\

Eigenfunction Property:\\
\tab $\mathcal{S}[e^{s_0t}] = H(s_0)e^{s_0t}$

One Sided Laplace Transform:\\
\tab $\ds F(s) = \mathcal{L}[f(t)u(t)] = \int_{0^-}^{\infty}f(t)e^{-st}\,dt$

Region of Convergence:\\
\begin{itemize}
	\itemsep0em
	\item Finite support $\rightarrow$ Entire s-plane.
	\item Causal function $\rightarrow$ $\sigma > \max(\sigma_i)$,  $-\infty < \omega < \infty$.
	\item Anti-causal $\rightarrow$ $\sigma < \min(\sigma_i)$,  $-\infty < \omega < \infty$.
	\item Non-causal $\rightarrow$ $\mathcal{R} = \mathcal{R}_\text{causal} \cap \mathcal{R}_\text{anti-causal}$.
\end{itemize}

Initial Value Theorem:\\
\tab $\ds f(0+) \Leftrightarrow \lim_{s\rightarrow\infty} sF(s)$

Final Value Theorem:\\
\tab $\ds \lim_{t\rightarrow\infty} f(t) \Leftrightarrow \lim_{s\rightarrow 0} sF(s)$

Bounded-Input Bounded-Output Stability:\\
\tab If the region of convergence contains the $j\omega$-axis, then the system is BIBO stable.

\section{Fourier Series}
\tab Fourier analysis in the steady state.

Eigenfunction Property:\\
\tab $\mathcal{S}[e^{j\omega_0t}] = H(j\omega_0)e^{j\omega_0t}$\\
\tab $\ds x(t) = \sum_k X_k e^{j\omega_kt} \implies y(t) = \sum_k X_k H(j\omega_k)e^{j\omega_kt}$\\\hspace{3.5cm}$\ds = \sum_k X_k \abs{H(j\omega_k)}e^{j(\omega_kt + \angle H(\omega_k))}$

Fourier Series Coefficients (for any $t_0$):\\
\tab $\ds X_k = \frac{1}{T_0}\int_{t_0}^{t_0 + T_0}x(t)e^{-jk\omega_0t}\,dt$

Parseval's Power Relation (for any $t_0$):\\
\tab $\ds P = \frac{1}{T_0} \int_{t_0}^{t_0 + T_0}\abs{x(t)}^2\,dt = \sum_{k = -\infty}^{\infty}\abs{X_k}^2$

Symmetry of Line Spectra:\\
\tab $\abs{X_k} = \abs{X_{-k}}$\\
\tab $\angle X_k = -\angle X_{-k}$

Trigonometric Fourier Series:\\
\tab $x(t) = X_0 + 2\sum_{k = 1}^{\infty}\abs{X_k}\cos(k\omega t + \Theta_k)$\\
\tab $x(t) = c_0 + 2\sum_{k = 1}^{\infty}[c_k\cos(k\omega_0 t) + d_k\sin(k\omega_0 t)]$\\
\tab $\ds c_k = \frac{1}{T_0} \int_{t_0}^{t_0 + T_0}x(t)\cos(k\omega_0 t)\,dt$ \quad $k$ = 0,1,2$\ldots$\\
\tab $\ds d_k = \frac{1}{T_0} \int_{t_0}^{t_0 + T_0}x(t)\sin(k\omega_0 t)\,dt$\quad $k$ = 1,2,3$\ldots$\\
\tab $\Theta_k = -\arctan(d_k/c_k)$
\\~\\
Fourier Coefficients from Laplace Transform:\\
\tab If $x_1(t)$ is a single period of $x(t)$, then\\
\tab $\ds X_k = \frac{1}{T_0}\mathcal{L}[x_1(t)] \big\rvert_{s = jk\omega_0}$

Response of LTI Systems to Periodic Signals:\\
\tab If the input to an LTI system has Fourier Series $x(t) = X_0 + 2\sum_{k=1}^{\infty}\abs{X_k}\cos(k\omega t + \angle X_k)$, then the steady state response is $y(t) = X_0\abs{H(j0)} + 2\sum_{k=1}^{\infty}\abs{X_k}\abs{H(jk\omega_0)}\cos(k\omega_0 t + \angle X_k + \angle H(jk\omega_0))$

\section{Fourier Transform}
\tab $\ds X(\omega) = \int_{-\infty}^{\infty}x(t)e^{-j\omega t}\,dt$\\
\tab $\ds x(t) = \frac{1}{2\pi}\int_{-\infty}^{\infty}X(\omega)e^{j\omega t}\,d\omega$

Fourier Transform from Laplace Transform (if $X(s)$ contains the $j\omega$-axis):\\
\tab $\mathcal{F}[x(t)] = \mathcal{L}[x(t)]\big\rvert_{s = j\omega} = X(s)\big\rvert_{s = j\omega}$

Duality:\\
\tab $\ds X(t) \xLeftrightarrow{\mathcal{F}} 2\pi x(-\omega)$

Fourier Transform of Periodic Signals:\\
\tab $\ds \sum_k X_k e^{jk\omega_0 t} \xLeftrightarrow{\mathcal{F}} \sum_k 2\pi_k \delta(\omega - k\omega_0)$

Parseval's Energy Relation:\\
\tab $\ds E = \int_{-\infty}^{\infty} \abs{x(t)}^2\,dt = \frac{1}{2\pi}\abs{X(\omega)}^2\,d\omega$

Symmetry of Spectral Representations:\\
\tab $\abs{X(\omega)} = \abs{X(-\omega)}$\\
\tab $\Re[X(\omega)] = \Re[X(-\omega)]$\\
\tab $\angle X(\omega) = -\angle X(-\omega)$\\
\tab $\Im[X(\omega)] = -\Im[X(-\omega)]$

% Footer content
\rule{0.3\linewidth}{0.25pt}
\scriptsize\\
Updated \today\\
\href{https://github.com/DonneyF/formula-sheets}{https://github.com/DonneyF/formula-sheets}
\end{multicols*}
\end{document}
